\documentclass{article}\usepackage{graphicx, color}
%% maxwidth is the original width if it is less than linewidth
%% otherwise use linewidth (to make sure the graphics do not exceed the margin)
\makeatletter
\def\maxwidth{ %
  \ifdim\Gin@nat@width>\linewidth
    \linewidth
  \else
    \Gin@nat@width
  \fi
}
\makeatother

\IfFileExists{upquote.sty}{\usepackage{upquote}}{}
\definecolor{fgcolor}{rgb}{0.2, 0.2, 0.2}
\newcommand{\hlnumber}[1]{\textcolor[rgb]{0,0,0}{#1}}%
\newcommand{\hlfunctioncall}[1]{\textcolor[rgb]{0.501960784313725,0,0.329411764705882}{\textbf{#1}}}%
\newcommand{\hlstring}[1]{\textcolor[rgb]{0.6,0.6,1}{#1}}%
\newcommand{\hlkeyword}[1]{\textcolor[rgb]{0,0,0}{\textbf{#1}}}%
\newcommand{\hlargument}[1]{\textcolor[rgb]{0.690196078431373,0.250980392156863,0.0196078431372549}{#1}}%
\newcommand{\hlcomment}[1]{\textcolor[rgb]{0.180392156862745,0.6,0.341176470588235}{#1}}%
\newcommand{\hlroxygencomment}[1]{\textcolor[rgb]{0.43921568627451,0.47843137254902,0.701960784313725}{#1}}%
\newcommand{\hlformalargs}[1]{\textcolor[rgb]{0.690196078431373,0.250980392156863,0.0196078431372549}{#1}}%
\newcommand{\hleqformalargs}[1]{\textcolor[rgb]{0.690196078431373,0.250980392156863,0.0196078431372549}{#1}}%
\newcommand{\hlassignement}[1]{\textcolor[rgb]{0,0,0}{\textbf{#1}}}%
\newcommand{\hlpackage}[1]{\textcolor[rgb]{0.588235294117647,0.709803921568627,0.145098039215686}{#1}}%
\newcommand{\hlslot}[1]{\textit{#1}}%
\newcommand{\hlsymbol}[1]{\textcolor[rgb]{0,0,0}{#1}}%
\newcommand{\hlprompt}[1]{\textcolor[rgb]{0.2,0.2,0.2}{#1}}%

\usepackage{framed}
\makeatletter
\newenvironment{kframe}{%
 \def\at@end@of@kframe{}%
 \ifinner\ifhmode%
  \def\at@end@of@kframe{\end{minipage}}%
  \begin{minipage}{\columnwidth}%
 \fi\fi%
 \def\FrameCommand##1{\hskip\@totalleftmargin \hskip-\fboxsep
 \colorbox{shadecolor}{##1}\hskip-\fboxsep
     % There is no \\@totalrightmargin, so:
     \hskip-\linewidth \hskip-\@totalleftmargin \hskip\columnwidth}%
 \MakeFramed {\advance\hsize-\width
   \@totalleftmargin\z@ \linewidth\hsize
   \@setminipage}}%
 {\par\unskip\endMakeFramed%
 \at@end@of@kframe}
\makeatother

\definecolor{shadecolor}{rgb}{.97, .97, .97}
\definecolor{messagecolor}{rgb}{0, 0, 0}
\definecolor{warningcolor}{rgb}{1, 0, 1}
\definecolor{errorcolor}{rgb}{1, 0, 0}
\newenvironment{knitrout}{}{} % an empty environment to be redefined in TeX

\usepackage{alltt}

\usepackage{color}
\definecolor{dkgreen}{rgb}{0,0.6,0}
\definecolor{gray}{rgb}{0.5,0.5,0.5}
\definecolor{mauve}{rgb}{0.58,0,0.82}
\usepackage[margin=1in]{geometry}
\usepackage{fancyhdr}
\pagestyle{fancy}
\lhead{\today}
\chead{Ying Qiao SID:21412301}
\rhead{Stat243 Fa12: Problem Set 2}
\lfoot{}
\cfoot{\thepage}
\rfoot{}
\usepackage{graphicx}
\usepackage{textcomp}
\usepackage{lmodern}
\usepackage[T1]{fontenc}
\usepackage{listings}

%% for inline R code: if the inline code is not correctly parsed, you will see a message
\newcommand{\rinline}[1]{SOMETHING WORNG WITH knitr}




\begin{document}


\section*{Problem 2}

After reading in the character vector text from the online traffic logs \textit{IPs.RData}, the 
code below could extract out all the IP numbers. In the meanwhile, it can determine how many IP 
addresses are in each element of the vector with function \textit{getIPnum}.

\begin{knitrout}
\definecolor{shadecolor}{rgb}{0.969, 0.969, 0.969}\color{fgcolor}\begin{kframe}
\begin{alltt}
patIP <- \hlstring{"((\textbackslash{}\textbackslash{}d\{1,3\}\textbackslash{}\textbackslash{}.)\{3\}\textbackslash{}\textbackslash{}d\{1,3\})"}  #Perl style pattern for IP
getIPnum <- \hlfunctioncall{function}(t) \{
    \hlfunctioncall{if} (\hlfunctioncall{length}(\hlfunctioncall{grep}(patIP, t, perl = TRUE)) == 0) \{
        \hlfunctioncall{return}(0)
    \} else \{
        \hlfunctioncall{return}(\hlfunctioncall{length}(\hlfunctioncall{gregexpr}(patIP, t, perl = TRUE)[[1]]))
    \}
\}  \hlcomment{# get number of IPs per text element}
getIPidx <- \hlfunctioncall{function}(t) \{
    \hlfunctioncall{return}(\hlfunctioncall{gregexpr}(patIP, t, perl = TRUE)[[1]])
\}  \hlcomment{# get index of IPs in text element}
getIPsub <- \hlfunctioncall{function}(i) \{
    \hlfunctioncall{return}(\hlfunctioncall{substring}(text[i], ipIdx[[i]], ipIdx[[i]] + \hlfunctioncall{attr}(ipIdx[[i]], \hlstring{"match.length"}) - 
        1))
\}  \hlcomment{# use idx to get IP strings}
ipNum <- \hlfunctioncall{sapply}(text, getIPnum, USE.NAMES = FALSE)  \hlcomment{# apply to the character vector}
ipIdx <- \hlfunctioncall{lapply}(text, getIPidx)  \hlcomment{# maintain list structure of index for each IP log}
ipStr <- \hlfunctioncall{sapply}(1:\hlfunctioncall{length}(text), getIPsub)
\hlcomment{# obtain IPs in list; multiple IPs per element stored in list items}
\end{alltt}
\end{kframe}
\end{knitrout}


The results of getting the number of IPs within one element of \textit{text} is one integer vector of 
the length \textit{length(text)}. All the extracted IPs are stored in the list \textit{ipStr}, 
with each element of a character vector in length \textit{ipNum[i]} of all the IPs from line \textit{text[i]}.
NA or no-IP results are treated as empty string.

\begin{knitrout}
\definecolor{shadecolor}{rgb}{0.969, 0.969, 0.969}\color{fgcolor}\begin{kframe}
\begin{verbatim}
## The results of # of IPs in each text element (1:100):

##   [1] 1 0 1 0 2 1 0 1 1 1 1 1 1 0 1 1 1 1 1 1 1 1 0 0 0 0 1 0 1 1 1 0 0 1 1
##  [36] 2 1 1 1 1 1 0 0 0 1 2 0 0 1 0 1 1 1 1 1 1 1 1 1 1 2 0 1 0 0 1 0 0 0 0
##  [71] 0 2 1 1 0 1 1 1 0 0 1 0 1 1 0 1 2 1 1 2 1 0 1 0 1 1 1 1 1 1

## The first 10 lines of results:

## $`from munnari.OZ.AU (localhost [127.0.0.1]) by delta.cs.mu.OZ.AU
## (8.11.6/8.11.6) with ESMTP id g7MBQPW13260; Thu, 22 Aug 2002 18:26:25
## +0700 (ICT)`
## [1] "127.0.0.1"
## 
## $`from SpoolDir by EMS-SRV0 (Mercury 1.44); 22 Aug 02 14:50:31 +0000`
## [1] ""
## 
## $`from ee.ed.ac.uk (sxs@dunblane [129.215.34.86]) by
## postbox.ee.ed.ac.uk (8.11.0/8.11.0) with ESMTP id g7ME1Li02942 for
## <forteana@yahoogroups.com>; Thu, 22 Aug 2002 15:01:21 +0100 (BST)`
## [1] "129.215.34.86"
## 
## $`from SpoolDir by EMS-SRV0 (Mercury 1.44); 22 Aug 02 15:01:34 +0000`
## [1] ""
## 
## $`from [192.168.0.4] (chello062178142216.4.14.vie.surfer.at
## [62.178.142.216]) (authenticated bits=0) by mail.uptime.at (8.12.5/8.12.5)
## with ESMTP id g7MEI7Vp022036 for
## <spamassassin-devel@lists.sourceforge.net>; Thu, 22 Aug 2002 16:18:07
## +0200`
## [1] "192.168.0.4"    "62.178.142.216"
## 
## $`from m206-56.dsl.tsoft.com ([198.144.206.56] helo=perkel.com) by
## darwin.ctyme.com with asmtp (TLSv1:RC4-MD5:128) (Exim 3.35 #1) id
## 17htgP-0004te-00; Thu, 22 Aug 2002 08:15:37 -0700`
## [1] "198.144.206.56"
## 
## $`by jlooney.jinny.ie (Postfix, from userid 500) id 4F57189D;
## Thu, 22 Aug 2002 16:25:45 +0100 (IST)`
## [1] ""
## 
## $`from dcu.ie (136.206.21.115) by hawk.dcu.ie (6.0.040) id
## 3D6203D3000136AD for iiu@taint.org; Thu, 22 Aug 2002 16:59:17 +0100`
## [1] "136.206.21.115"
## 
## $`from [66.218.67.174] by n19.grp.scd.yahoo.com with NNFMP;
## 22 Aug 2002 16:11:27 -0000`
## [1] "66.218.67.174"
## 
## $`from [66.218.67.189] by n10.grp.scd.yahoo.com with NNFMP;
## 22 Aug 2002 16:17:40 -0000`
## [1] "66.218.67.189"
\end{verbatim}
\end{kframe}
\end{knitrout}



\end{document}
