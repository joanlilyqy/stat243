\documentclass{article}
\usepackage[margin=1in]{geometry}
\usepackage{fancyhdr}
\pagestyle{fancy}
\lhead{\today}
\chead{}
\rhead{Stat243 Fa12}
\lfoot{}
\cfoot{\thepage}
\rfoot{}
\usepackage{amsmath,amsthm,amssymb}
\usepackage{graphicx}
%% for inline R code: if the inline code is not correctly parsed, you will see a message
\newcommand{\rinline}[1]{SOMETHING WORNG WITH knitr}
%% begin.rcode setup, include=FALSE
% opts_chunk$set(fig.path='figure/latex-', cache.path='cache/latex-')
%% end.rcode

\title{Stat243: Problem Set 1}
\author{Ying Qiao}

\begin{document}
\maketitle
\newpage

\section*{Problem 1}
\subsection*{(a)}

\subsection*{(b)}

\subsection*{(c)}

\newpage

\section*{Problem 2}
\subsection*{(a)}

\subsection*{(b)}

\newpage

\section*{Problem 3}

Crime rates in the US are high compared to European countries. Here I `analyze' the variation in murder across US states using R. I show a histogram of rates of arrest for murder for the 50 states and find the states with the lowest and highest murder arrest rates.

%% begin.rcode arrest-hist, dev='cairo_pdf', fig.width=5, fig.height=5, out.width='.8\\textwidth'
% hist(USArrests$Murder)
%% end.rcode

%% chunk options: cache this chunk
%% begin.rcode arrest-cache, cache=TRUE
% lowHi <- c(which.min(USArrests$Murder), which.max(USArrests$Murder))
% attributes(USArrests)$row.names[lowHi]
%% end.rcode

The state with the lowest rate is \rinline{attributes(USArrests)$row.names[which.min(USArrests$Murder)]}. The state with the highest rate is \rinline{attributes(USArrests)$row.names[which.max(USArrests$Murder)]}.

\end{document}