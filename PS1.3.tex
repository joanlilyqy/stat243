\documentclass{article}

\usepackage{color}
\definecolor{dkgreen}{rgb}{0,0.6,0}
\definecolor{gray}{rgb}{0.5,0.5,0.5}
\definecolor{mauve}{rgb}{0.58,0,0.82}
\usepackage[margin=1in]{geometry}
\usepackage{fancyhdr}
\pagestyle{fancy}
\lhead{\today}
\chead{Ying Qiao SID:21412301}
\rhead{Stat243 Fa12: Problem Set 1}
\lfoot{}
\cfoot{\thepage}
\rfoot{}
\usepackage{graphicx}
\usepackage{textcomp}
\usepackage{lmodern}
\usepackage[T1]{fontenc}
\usepackage{listings}
\lstset{
	frame=tb,
	language=sh,
	aboveskip=3mm,
	belowskip=3mm,
	basicstyle={\small\ttfamily},
	numberstyle=\tiny\color{gray},
	keywordstyle=\color{blue},
	commentstyle=\color{dkgreen},
	stringstyle=\color{mauve},
	breaklines=true,
	breakatwhitespace=true,
	prebreak=\textbackslash,
	showstringspaces=false,
	columns=fullflexible,
	upquote=true
	}
%% for inline R code: if the inline code is not correctly parsed, you will see a message
\newcommand{\rinline}[1]{SOMETHING WORNG WITH knitr}
%% begin.rcode setup, include=FALSE
% opts_chunk$set(fig.path='figure/latex-', cache.path='cache/latex-')
%% end.rcode

\immediate\write18{shelltex PS1.1}
%%\immediate\write18{./PS1.2.sh > PS1.2.tex}

\begin{document}

\section*{Problem 1}
\subsection*{(a)}

To extract the data for Republican and Democratic candidates of Senate into two files,
\textit{wget} is used to download the summary file and piping with \textit{grep} to file output gives me the result.

\begin{lstlisting}
# Download .csv from FEC
wget -q -O "CandidateSummary.csv" "http://www.fec.gov/data/CandidateSummary.do?format=csv"
# Extract Senate, and then split into 'REP' and 'DEM'
grep "\"S\"" CandidateSummary.csv | grep "\"REP\"" > REP.csv
grep "\"S\"" CandidateSummary.csv | grep "\"DEM\"" > DEM.csv
\end{lstlisting}

With pre-processing using \textit{sed} commands given, we can then \textit{cut} the fields and
\textit{sort} according to total contributions by printing the top 5 with \textit{head}.

\begin{lstlisting}
# Preprocessing before spliting fields
sed 's/\([^",]\),/\1/g' REP.csv | sed 's/[$"]//g' > REPclean.csv
sed 's/\([^",]\),/\1/g' DEM.csv | sed 's/[$"]//g' > DEMclean.csv
# Sort on total contribution and display the richest five
cut -d ',' -f 2,3,4,5,7,8,20 REPclean.csv | sort -n -r -t ',' -k 7 | head -n 5
cut -d ',' -f 2,3,4,5,7,8,20 DEMclean.csv | sort -n -r -t ',' -k 7 | head -n 5
\end{lstlisting}

\subsection*{(b)}

After downloading the files, we \textit{unzip} to get the \textit{*.txt} files. With stored \textit{PCID},
we can extract the number of contributions for a given candidate. This part of work is focused on the current
candidates for presidential race year 2012.

\begin{lstlisting}
# Download from FEC
wget -q "ftp://ftp.fec.gov/FEC/2012/cn12.zip"
wget -q "ftp://ftp.fec.gov/FEC/2012/indiv12.zip"
unzip -q cn12    # get cn.txt
unzip -q indiv12 # get itcont.txt
# Clear cn.txt down to current candidates for presidential race year 2012
cut cn.txt -d '|' -f 1-6,9-11 | grep "|2012|US|P|C|" > pres_cn.txt
# Grab PC ID with last name, return the first one if duplicated
NAME="OBAMA"
PCID=$(grep "|${NAME}," pres_cn.txt | cut -d '|' -f 8 | head -n 1)
# Individual contribution counts (only over 200 is recorded, ingore negative number counts, treat as input error)
# National counts
grep "${PCID}" itcont.txt | cut -d '|' -f 15 | wc -l
# California counts
grep "${PCID}" itcont.txt | grep "CA" | cut -d '|' -f 15 | wc -l
\end{lstlisting}

\subsection*{(c)}

We functionalize the shell commands in (b) to get results for multiple candidates. A file of candidates names is generated for testing the functions.

\begin{lstlisting}
# Get the last name(s) of the 2012 presidential candidates in the REP and DEM
# These are candidates of interest for my discussion
grep "|DEM\|REP|" pres_cn.txt | cut -d '|' -f 2 | cut -d ',' -f 1 > name_cn.txt
# Clean for split fields to get total contributions of candidates
sed 's/\([^",]\),/\1/g' CandidateSummary.csv | sed 's/[$"]//g' > CanSum_clean.csv
\end{lstlisting}

Three functions are built for getting the total contributions (\textit{getTotalCont}), number of contributions 
(over \$200) nationwide (\textit{getContNation}) and in California (\textit{getContCA}).

\begin{lstlisting}
# Functions
function getTotalCont () {
    ID=$(grep "|${1}," pres_cn.txt | cut -d '|' -f 1 | head -n 1)
    NM=$(grep "${ID}" pres_cn.txt | cut -d '|' -f 2)
    TOTAL=$(grep "${ID}" CanSum_clean.csv | cut -d ',' -f 20)
}
function getContNation () {
    ID=$(grep "|${1}," pres_cn.txt | cut -d '|' -f 8 | head -n 1)
    NM=$(grep "${ID}" pres_cn.txt | cut -d '|' -f 2)
    NCOUNT=$(grep "${ID}" itcont.txt | cut -d '|' -f 15 | wc -l)
}
function getContCA () {
    ID=$(grep "|${1}," pres_cn.txt | cut -d '|' -f 8 | head -n 1)
    NM=$(grep "${ID}" pres_cn.txt | cut -d '|' -f 2)
    CACOUNT=$(grep "${ID}" itcont.txt | grep "CA" | cut -d '|' -f 15 | wc -l)
}
\end{lstlisting}

The loop-to-loop calling of the functions for multiple predefined candidates is realized with \textit{for...do...done}.

\begin{lstlisting}
# Call the function
cnName=$(cat name_cn.txt)
for name in $cnName
do
    getTotalCont $name
    getContNation $name
    getContCA $name
    echo "${NM} :"
    echo "${TOTAL}|${NCOUNT}|${CACOUNT}"
done
\end{lstlisting}

\subsection*{Shell script running results}

\input{"./PS1.1.tex"}


\newpage

\section*{Problem 2}
\subsection*{(a)}

The shell function \textit{remoteRJobs} is realized with \textit{ps} remotely after \textit{ssh} to the
specified machine. The jobs are sorted according to CPU usage percentage and will be displayed with the given number of lines. The default is to display all the jobs.

\begin{lstlisting}
# Find R jobs and CPU usage on remote machines
function remoteRJobs () {
		if [ $# == "2" ]
		then 
		   echo "The top ${2} %CPU usage of R jobs running on ${1}:" 
		   echo "PID	UID	%CPU	CMD"
		   ssh $1 ps -C R -o pid,user,%cpu,comm --sort=-%cpu | grep -v PID | head -n $2
		   echo ""
		elif [ $# == "1" ]
		then 
		   echo "The %CPU usage of R jobs running on ${1}:" 
		   echo "PID	UID	%CPU	CMD"
		   ssh $1 ps -C R -o pid,user,%cpu,comm --sort=-%cpu | grep -v PID
		   echo ""
		else
		   echo "ERROR: ***************************"
		   echo "remoteRJobs MACHINE [number]"
		   echo ""
		fi 
}
\end{lstlisting}


\subsection*{(b)}

For extra function on the \textit{remoteRJobs}, the 2.0 version utilizes a \textit{mysum} function for summing up the CPU and MEM usage. The calculation is done purely in shell without porting to R. Also the total CPU usage is averaged to the number of cores on the given remote machine, while the MEM usage is just simple sum up.

\begin{lstlisting}
# Extend 2.(a) to add up the CPU and memory use of all R jobs
function mysum () {
		sum=0
		for num in $(cat $1)
		do
		    sum=$((sum+=num))
		done
}

function remoteRJobs2 () {
		if [ $# == "2" ]
		then 
		   echo "The top ${2} %CPU usage of R jobs running on ${1}:" 
		   echo "PID	UID	%CPU	%MEM	CMD"
		   ssh $1 ps -C R -o pid,user,pcpu,pmem,comm --sort=-pcpu | grep -v "%CPU" | head -n $2
		   
		   cpunum=$(ssh $1 grep processor /proc/cpuinfo | wc -l)
		   # calculate cpu and memory usage
		   ssh $1 ps -C R -o pcpu --sort=-pcpu | grep -v "%CPU" | head -n $2 | sed 's/ //' | sed 's/\.//' | sed 's/^0//' > cpu.txt
		   ssh $1 ps -C R -o pmem --sort=-pmem | grep -v "%MEM" | head -n $2 | sed 's/ //' | sed 's/\.//' | sed 's/^0//' > mem.txt
		   mysum cpu.txt;sum=$((sum/=10));sum=$((sum/=cpunum))
		   echo "The top ${2} total %CPU used by R jobs on ${1} is ${sum}% for ${cpunum} CPUs"
		   mysum mem.txt;sum=$((sum/=10))
		   echo "The top ${2} total %MEM used by R jobs on ${1} is ${sum}%"
		   
		   echo ""
		elif [ $# == "1" ]
		then 
		   echo "The %CPU usage of R jobs running on ${1}:" 
		   echo "PID	UID	%CPU	%MEM	CMD"
		   ssh $1 ps -C R -o pid,user,pcpu,pmem,comm --sort=-pcpu | grep -v "%CPU"
		   
		   cpunum=$(ssh $1 grep processor /proc/cpuinfo | wc -l)
		   # calculate cpu and memory usage
		   ssh $1 ps -C R -o pcpu --sort=-pcpu | grep -v "%CPU" | sed 's/ //' | sed 's/\.//' | sed 's/^0//' > cpu.txt
		   ssh $1 ps -C R -o pmem --sort=-pmem | grep -v "%MEM" | sed 's/ //' | sed 's/\.//' | sed 's/^0//' > mem.txt
		   mysum cpu.txt;sum=$((sum/=10));sum=$((sum/=cpunum))
		   echo "The total %CPU used by R jobs on ${1} is ${sum}% for ${cpunum} CPUs"
		   mysum mem.txt;sum=$((sum/=10))
		   echo "The total %MEM used by R jobs on ${1} is ${sum}%"
		   
		   echo ""
		else
		   echo "ERROR: ***************************"
		   echo "remoteRJobs MACHINE [number]"
		   echo ""
		fi 
}
\end{lstlisting}

\subsection*{Shell script running results}

\input{"./PS1.2.tex"}


\newpage

\section*{Problem 3}

Crime rates in the US are high compared to European countries. Here I `analyze' the variation in murder across US states using R. I show a histogram of rates of arrest for murder for the 50 states and find the states with the lowest and highest murder arrest rates.

%% begin.rcode arrest-hist, dev='cairo_pdf', fig.width=5, fig.height=5, out.width='.8\\textwidth'
% hist(USArrests$Murder)
%% end.rcode

%% chunk options: cache this chunk
%% begin.rcode arrest-cache, cache=TRUE
% lowHi <- c(which.min(USArrests$Murder), which.max(USArrests$Murder))
% attributes(USArrests)$row.names[lowHi]
%% end.rcode

The state with the lowest rate is \rinline{attributes(USArrests)$row.names[which.min(USArrests$Murder)]}. The state with the highest rate is \rinline{attributes(USArrests)$row.names[which.max(USArrests$Murder)]}.

\end{document}
