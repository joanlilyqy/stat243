\documentclass{article}

\usepackage{color}
\definecolor{dkgreen}{rgb}{0,0.6,0}
\definecolor{gray}{rgb}{0.5,0.5,0.5}
\definecolor{mauve}{rgb}{0.58,0,0.82}
\usepackage[margin=1in]{geometry}
\usepackage{fancyhdr}
\pagestyle{fancy}
\lhead{\today}
\chead{Ying Qiao SID:21412301}
\rhead{Stat243 Fa12: Problem Set 2}
\lfoot{}
\cfoot{\thepage}
\rfoot{}
\usepackage{graphicx}
\usepackage{textcomp}
\usepackage{lmodern}
\usepackage[T1]{fontenc}
\usepackage{listings}

%% for inline R code: if the inline code is not correctly parsed, you will see a message
\newcommand{\rinline}[1]{SOMETHING WORNG WITH knitr}
%% begin.rcode setup, include=FALSE
% opts_chunk$set(fig.path='figure/latex-', cache.path='cache/latex-', width=90)
%% end.rcode


\begin{document}
%% begin.rcode echo=FALSE
%% rm(list=ls(all=TRUE)) #remove all objects
%% end.rcode
\section*{Problem 1}

To have a function that lists the objects in the workspace in a nicely formatted manner, serveral auxiliary 
printing function (\textit{printObj}, \textit{printUnit}) is created first for cleaner coding. 

%% begin.rcode PS21-print, cache=TRUE, results="hide"
%%## Aux Printing Fun:
%%## Define printing control global parameter
%%wid = 20		# object name priting width
%%digit = 10	# byte size printing digits
%%## Print out file size according to auto-match units
%%printUnit <- function (obj) { 
%%            	units = c("", "k", "M", "G")
%%            	i = 1; while (obj >= 1024) { obj = obj / 1024; i = i + 1;} # get proper units
%%            	if (i == 1) { cat(format(obj,width = digit,justify = 'right'),"\n",sep = "") }
%%            	else { cat(format(as.integer(obj),width = digit - 1,justify = 'right'),units[i],"\n",sep = "") }
%%            }
%%## Print out the object list in a nicely formatted way																
%%printObj <- function (objList, objName, pretty) { 
%%            	cat(format("object",width = wid),format("bytes",width = digit,justify = 'right'),"\n",sep = "")
%%            	if (pretty == FALSE){ # pretty==TRUE: it will call the printUnit to print out proper units for sizes	
%%								t<-sapply(1:length(objList),function(i){cat(format(objName[i],width=wid),format(objList[i],width=digit,justify='right'),"\n",sep="")})}
%%            	else{ # default:      it will just print out number of bytes	
%%                t<-sapply(1:length(objList),function(i){cat(format(names(objList)[i], width = wid));printUnit(objList[i])})}
%%            }
%% end.rcode

The main function will list objects ordered by size from largest to smallest; it allows the user to 
either specify how many objects to list or to specify that objects larger than a certain size 
to be listed. The argument \textit{pretty}, when TRUE, results in the size of the objects being printed 
with the units tailored by function \textit{printUnit}.
By default, \textit{lsObj<-function(numls=100, sizeB=0, pretty=FALSE)} 
will give sensible results.	
Also, the main function shown below is already written to be able to print out only
objects in the frame of the function call when called from within another function.

\newpage
%% begin.rcode PS21-main, cache=TRUE, results="hide"
%%lsObj <- function (numls = 100, sizeB = 0, pretty = FALSE) {	
%%					listLs <- ls(envir = parent.frame()) # object list
%%         	sizeLs <- sapply(listLs,function(x){object.size(get(x,envir = parent.frame(n = 4)))}) 
%%         	# get all object sizes from within the called frame; n=4:FUN->sapply->lsObj->call(lsObj)
%%					tmpEnv <- NULL; nameEnv <- NULL # store object sizes and names within a certain user-defined ENV
%%					tmpFun <- NULL; nameFun <- NULL # store object sizes and names within a function closure
%%					for (item in listLs){ # elborate objects to find hidden ones
%%					  objitem <- get(item, envir = parent.frame())
%%						if (is.environment(objitem)) { # get sizes and store names with ENV prefix
%%							tt <- sapply(ls(envir = objitem), function(x){object.size(get(x,envir = objitem))})
%%							tmpEnv <- c(tmpEnv,tt); nameEnv <- c(nameEnv, paste(item,"$",ls(envir = objitem),sep = ""))
%%						}
%%						if (is.function(objitem) && identical(environment(objitem), parent.frame())== FALSE ){ # get sizes and store names with FUN prefix
%%							tt <- sapply(ls(envir = environment(objitem)), function(x){object.size(get(x,envir = environment(objitem)))})
%%							tmpFun <- c(tmpFun,tt); nameFun <- c(nameFun, paste(item,"$",ls(envir = environment(objitem)),sep = ""))
%%						}
%%					}
%%					names(tmpEnv) <- nameEnv; names(tmpFun) <- nameFun
%%         	if (length(tmpEnv) > 0) { sizeLs <- c(sizeLs,tmpEnv) } #if there is use-defined ENV
%%        	if (length(tmpFun) > 0) { sizeLs <- c(sizeLs,tmpFun) } #if there is FUN closure
%%         	sizeLs <- sizeLs[sizeLs[] > sizeB] # find objects larger than sizeB(bytes)
%%         	if (numls > length(sizeLs)) { numls = length(sizeLs) } #prune incorrect listing number
%%         	sizeLs <- sort(sizeLs, decreasing = TRUE)[1:numls] # find the numls largest objects
%%         	nameLs <- names(sizeLs) # get the name of vector
%%          printObj(sizeLs,nameLs,pretty) # print using the aux functions
%%         }
%% end.rcode

\newpage

The above main function has the capability to deal with user-created environments and objects within a closure.
It elborates the objects obtained from the \textit{ls()} function to see if there are user-defined ENV or FUN closure.
When detected, it then goes one frame down to the enclosing environment to get the hidden objects and assign names to
them with ENV/FUN prefix.


To create objects of various sizes for testing, I generated vectors of random normals of different lengths, 
along with serveral hidden ENV/FUN examples for testing.


%% begin.rcode PS21-test0, cache=TRUE, eval=FALSE, results="hide"
%%#### Generate objects test cases
%%ob0 <- rnorm(1);ob1 <- rnorm(10);ob2 <- rnorm(100);ob3 <- rnorm(1000);ob4 <- rnorm(10000);ob5 <- rnorm(100000)
%%ob6 <- rnorm(1000000);ob7 <- rnorm(10000000)
%% end.rcode

%% begin.rcode PS21-test1, cache=TRUE, echo=FALSE, results="markup"
%%#### Generate objects test cases
%%ob0 <- rnorm(1);ob1 <- rnorm(10);ob2 <- rnorm(100);ob3 <- rnorm(1000);ob4 <- rnorm(10000);ob5 <- rnorm(100000)
%%ob6 <- rnorm(1000000);ob7 <- rnorm(10000000)
%%#### Test case
%%cat("**********Test Case 1*************\nlsObj(5,128)\n")
%%lsObj(5,128)
%%cat("**********Test Case 2*************\nlsObj(sizeB = 1024)\n")
%%lsObj(sizeB = 1024)
%%cat("**********Test Case 3*************\nlsObj(numls = 3)\n")
%%lsObj(numls = 3)
%%cat("**********Test Case 4*************\nlsObj(pretty = TRUE)\n")
%%lsObj(pretty = TRUE)
%% end.rcode

%% begin.rcode PS21-test2, cache=TRUE, eval=FALSE, results="hide"
%%testFunc <- function () {
%%            	data <- rnorm(10000);	data2 <- rnorm(100000)
%%            	myFun <- function(theta) { 
%%                       dist <- rnorm(theta); return(exp(dist / theta)); }
%%	lsObj()}
%%testEx <- function () {
%%          	x <- rnorm(100000);	data <- rnorm(1000)
%%          	e <- new.env(); e$a <- x # an object hidden in an environment
%%          	e2 <- new.env(); e2$a2 <- x; e2$b2 <- 52
%%          	myFun <- function(theta) { 
%%                     dist <- rnorm(theta); return(exp(dist / theta)); }
%%          	myFun1 <- function(theta) { 
%%                      dist <- rnorm(theta); return(exp(dist / theta)); }
%%          	myFun2 <- with(list( data = x ), # an object hidden in a closure
%%                          function(theta) { 
%%                          dist <- rdist(data); return(exp(dist / theta)); } )
%%            myFun3 <- with(list( data2 = x ),
%%                          function(theta2) { 
%%                          dist <- rdist(data); return(exp(dist / theta2)); } )
%%	lsObj()}
%% end.rcode

%% begin.rcode PS21-test3, cache=TRUE, echo=FALSE, results="markup"
%%cat("**********Test Case 5*************\nlsObj() in testFunc()\n")
%%testFunc <- function () {
%%            	data <- rnorm(10000)
%%            	data2 <- rnorm(100000)
%%            	myFun <- function(theta) { dist <- rnorm(theta); return(exp(dist / theta)); }
%%	lsObj()
%%}
%%testFunc()
%%cat("**********Test Case 6*************\nlsObj() in test testEx()\n")
%%testEx <- function () {
%%          	x <- rnorm(100000)
%%          	data <- rnorm(1000)
%%          	e <- new.env(); e$a <- x # an object hidden in an environment
%%          	e2 <- new.env(); e2$a2 <- x; e2$b2 <- 52
%%          	myFun <- function(theta) { dist <- rnorm(theta); return(exp(dist / theta)); }
%%          	myFun1 <- function(theta) { dist <- rnorm(theta); return(exp(dist / theta)); }
%%          	myFun2 <- with(list( data = x ), # an object hidden in a closure
%%                          function(theta) { dist <- rdist(data); return(exp(dist / theta)); } )
%%            myFun3 <- with(list( data2 = x ),
%%                          function(theta2) { dist <- rdist(data); return(exp(dist / theta2)); } )
%%	lsObj()
%%}
%%testEx()
%% end.rcode


\end{document}
