\documentclass{article}

\usepackage{color}
\definecolor{dkgreen}{rgb}{0,0.6,0}
\definecolor{gray}{rgb}{0.5,0.5,0.5}
\definecolor{mauve}{rgb}{0.58,0,0.82}
\usepackage[margin=1in]{geometry}
\usepackage{fancyhdr}
\pagestyle{fancy}
\lhead{\today}
\chead{Ying Qiao SID:21412301}
\rhead{Stat243 Fa12: Problem Set 2}
\lfoot{}
\cfoot{\thepage}
\rfoot{}
\usepackage{graphicx}
\usepackage{textcomp}
\usepackage{lmodern}
\usepackage[T1]{fontenc}
\usepackage{listings}

%% for inline R code: if the inline code is not correctly parsed, you will see a message
\newcommand{\rinline}[1]{SOMETHING WORNG WITH knitr}
%% begin.rcode setup, include=FALSE
% opts_chunk$set(fig.path='figure/latex-', cache.path='cache/latex-')
%% end.rcode


\begin{document}
%% begin.rcode echo=FALSE
%% rm(list=ls(all=TRUE)) #remove all objects
%% end.rcode
\section*{Problem 2}

After reading in the character vector text from the online traffic logs \textit{IPs.RData}, the 
code below could extract out all the IP numbers. In the meanwhile, it can determine how many IP 
addresses are in each element of the vector with function \textit{getIPnum}.

%% begin.rcode PS22-regex0, cache=TRUE, eval=FALSE, results="hide"
%% patIP <- "((\\d{1,3}\\.){3}\\d{1,3})" #Perl style pattern for IP
%% getIPnum <- function (t) { if (length(grep(patIP,t,perl=TRUE)) == 0) {return (0)}
%%                           else {return (length(gregexpr(patIP,t,perl=TRUE)[[1]]))}
%%                         } # get number of IPs per text element
%% getIPidx <- function (t) { return (gregexpr(patIP,t,perl=TRUE)[[1]]) } # get index of IPs in text element
%% getIPsub <- function (i) { return (substring(text[i],ipIdx[[i]],ipIdx[[i]]+attr(ipIdx[[i]],"match.length")-1))} # use idx to get IP strings
%% ipNum <- sapply(text, getIPnum, USE.NAMES = FALSE) # apply to the character vector
%% ipIdx <- lapply(text, getIPidx) # maintain list structure of index for each IP log
%% ipStr <- sapply(1:length(text), getIPsub) 
%% # obtain IPs in list; multiple IPs per element stored in list items
%% end.rcode

The results of getting the number of IPs within one element of \textit{text} is one integer vector of 
the length \textit{length(text)}. All the extracted IPs are stored in the list \textit{ipStr}, 
with each element of a character vector in length \textit{ipNum[i]} of all the IPs from line \textit{text[i]}.
NA or no-IP results are treated as empty string.

%% begin.rcode PS22-regex1, cache=TRUE, echo=FALSE, results="markup"
%% rm(list=ls(all=TRUE)) # remove all objects
%% load('IPs.RData') # import text data
%% patIP <- "((\\d{1,3}\\.){3}\\d{1,3})" #Perl style pattern for IP
%% getIPnum <- function (t) { if (length(grep(patIP,t,perl=TRUE)) == 0) {return (0)}
%%                           else {return (length(gregexpr(patIP,t,perl=TRUE)[[1]]))}
%%                         } # get number of IPs per text element
%% getIPidx <- function (t) { return (gregexpr(patIP,t,perl=TRUE)[[1]]) } # get index of IPs in text element
%% getIPsub <- function (i) { return (substring(text[i],ipIdx[[i]],ipIdx[[i]]+attr(ipIdx[[i]],"match.length")-1))} # use idx to get IP strings
%% ipNum <- sapply(text, getIPnum, USE.NAMES = FALSE) # apply to the character vector
%% ipIdx <- lapply(text, getIPidx) # maintain list structure of index for each IP log
%% ipStr <- sapply(1:length(text), getIPsub) 
%% names(ipStr) <- text
%% cat("The results of # of IPs in each text element (1:100):\n")
%% print(ipNum[1:100])
%% cat("The first 10 lines of results:\n")
%% print(head(ipStr, n= 10))
%% end.rcode


\end{document}
