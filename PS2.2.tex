\documentclass{article}

\usepackage{color}
\definecolor{dkgreen}{rgb}{0,0.6,0}
\definecolor{gray}{rgb}{0.5,0.5,0.5}
\definecolor{mauve}{rgb}{0.58,0,0.82}
\usepackage[margin=1in]{geometry}
\usepackage{fancyhdr}
\pagestyle{fancy}
\lhead{\today}
\chead{Ying Qiao SID:21412301}
\rhead{Stat243 Fa12: Problem Set 2}
\lfoot{}
\cfoot{\thepage}
\rfoot{}
\usepackage{graphicx}
\usepackage{textcomp}
\usepackage{lmodern}
\usepackage[T1]{fontenc}
\usepackage{listings}

%% for inline R code: if the inline code is not correctly parsed, you will see a message
\newcommand{\rinline}[1]{SOMETHING WORNG WITH knitr}
%% begin.rcode setup, include=FALSE
% opts_chunk$set(fig.path='figure/latex-', cache.path='cache/latex-')
%% end.rcode


\begin{document}

\section*{Problem 1}


1. Write an R function that lists the objects in your workspace, ordered by size from largest to smallest,
in a nicely formatted manner. For clean code, you may need to write auxiliary functions that are called
by your main function.
(a) Your function should allow the user to either specify how many objects to list or to specify that
all objects larger than a certain size be listed. By default, it should do something sensible.
(b) Your function should include an argument, called ��pretty��, that when TRUE results in the size
of the objects being printed with the units used tailored to the size of the object. When FALSE,
all sizes should be in bytes, but still nicely formatted. Here��s an example of part of the desired
output of pretty = TRUE:
object bytes
y 763M
Onomatopoeia 234k
z 30k
z 128
x2 96
1
(c) When called from within another function, your function should work and should print out only
objects in the frame of that function call.


(d) Extra credit: Sometimes user-created objects can be hidden from view. Two examples are objects
within user-created environments and objects within something called a closure.
x <- rnorm(100000)
e <- new.env(); e$a <- x # an object hidden in an environment
myFun <- with(list( data = x ), # an object hidden in a closure
function(theta) {
dist <- rdist(data)
return(exp(dist / theta))
})
The closure is a way to associate data with a function so that it doesn��t have to get passed in as
an argument.
Task: Make sure that your function can handle both these cases - i.e., that objects within environments
or closures are included in the output. The environment() and is.environment() functions
are likely to be helpful.

To create objects of various sizes for testing, I usually just generate vectors of
random normals of different lengths, remembering that a vector of numerics has 8 bytes per element.

\subsection*{(a)}

To 
\textit{wget} 


%% begin.rcode PS21-cache, cache=TRUE





%% end.rcode


With pre-processing using \textit{sed} commands given, we can then \textit{cut} the fields and
\textit{sort} according to total contributions by printing the top 5 with \textit{head}.

\begin{lstlisting}

\end{lstlisting}

\subsection*{(b)}

After downloading the files, we \textit{unzip} to get the \textit{*.txt} files. With stored \textit{PCID},
we can extract the number of contributions for a given candidate. This part of work is focused on the current
candidates for presidential race year 2012.

\begin{lstlisting}

\end{lstlisting}

\subsection*{(c)}

We functionalize the shell commands in (b) to get results for multiple candidates. A file of candidates names is generated for testing the functions.

\begin{lstlisting}


\end{lstlisting}

The loop-to-loop calling of the functions for multiple predefined candidates is realized with \textit{for...do...done}.

\begin{lstlisting}

\end{lstlisting}

\subsection*{Shell script running results}




\newpage

\section*{Problem 2}
\subsection*{(a)}

The shell function \textit{remoteRJobs} is realized with \textit{ps} remotely after \textit{ssh} to the
specified machine. The jobs are sorted according to CPU usage percentage and will be displayed with the given number of lines. The default is to display all the jobs.

\begin{lstlisting}

\end{lstlisting}


\subsection*{(b)}

For extra function on the \textit{remoteRJobs}, the 2.0 version utilizes a \textit{mysum} function for summing up the CPU and MEM usage. The calculation is done purely in shell without porting to R. Also the total CPU usage is averaged to the number of cores on the given remote machine, while the MEM usage is just simple sum up.

\begin{lstlisting}

}
\end{lstlisting}

\subsection*{Shell script running results}




\newpage

\section*{Problem 3}

Crime rates in the US are high compared to European countries. Here I `analyze' the variation in murder across US states using R. I show a histogram of rates of arrest for murder for the 50 states and find the states with the lowest and highest murder arrest rates.

%% begin.rcode arrest-hist, dev='cairo_pdf', fig.width=5, fig.height=5, out.width='.8\\textwidth'
% hist(USArrests$Murder)
%% end.rcode

%% begin.rcode arrest-cache, cache=TRUE
% lowHi <- c(which.min(USArrests$Murder), which.max(USArrests$Murder))
% attributes(USArrests)$row.names[lowHi]
%% end.rcode

The state with the lowest rate is \rinline{attributes(USArrests)$row.names[which.min(USArrests$Murder)]}. The state with the highest rate is \rinline{attributes(USArrests)$row.names[which.max(USArrests$Murder)]}.

\end{document}
