\documentclass{article}

\usepackage{color}
\definecolor{dkgreen}{rgb}{0,0.6,0}
\definecolor{gray}{rgb}{0.5,0.5,0.5}
\definecolor{mauve}{rgb}{0.58,0,0.82}
\usepackage[margin=1in]{geometry}
\usepackage{fancyhdr}
\pagestyle{fancy}
\lhead{\today}
\chead{Ying Qiao SID:21412301}
\rhead{Stat243 Fa12: Problem Set 5}
\lfoot{}
\cfoot{\thepage}
\rfoot{}
\usepackage{graphicx}
\usepackage{textcomp}
\usepackage{lmodern}
\usepackage[T1]{fontenc}
\usepackage{listings}

%% for inline R code: if the inline code is not correctly parsed, you will see a message
\newcommand{\rinline}[1]{SOMETHING WORNG WITH knitr}
%% begin.rcode setup, include=FALSE
% opts_chunk$set(fig.path='figure/latex-', cache.path='cache/latex-', width=90)
%% end.rcode


\begin{document}
\section*{Problem 2}

\subsubsection*{(a)}

\hspace{12 pt} Considering the adaptive rejection sampling (ARS) algorithm, we build a set of $p$ points,
$x_1,...,x_p$ and an upper envelope function that is linear on the log scale withing each interval, i.e.
$f(x) = exp(a_i+b_ix)$ for $x\in(x_{i-1},x_i)$

We need to normalize this exponentiated upper envelop function in order to have a density of interest that
we can draw samples from.\newline
The normalization factor $N_f$ is constant,\newline
$N_f = \displaystyle\sum_{i=1}^p \int_{x_{i-1}}^{x_i} exp(a_i+b_ix)dx 
     = \displaystyle\sum_{i=1}^p \frac{e^{a_i}}{b_i}[e^{b_ix_i}-e^{b_ix_{i-1}}]$.\newline
The probability associated with each interval $P_i$ is, \newline
$P_i = Prob(x\in(x_{i-1},x_i)) = \frac{1}{N_f}\displaystyle\int_{x_{i-1}}^{x_i}exp(a_i+b_ix)dx
     = \frac{e^{a_i}}{N_fb_i}[e^{b_ix_i}-e^{b_ix_{i-1}}]$.\newline


\subsubsection*{(b)}

\hspace{12 pt} We want to draw $x$ from the $i$th interval using the inverse CDF method.\newline
The CDF function $F(x')$ is,\newline
$F(x') = Prob(x\in(x_{i-1},x')) = \frac{1}{N_f}\displaystyle\int_{x_{i-1}}^{x'}exp(a_i+b_ix)dx
        = \frac{e^{a_i}}{N_fb_i}[e^{b_ix'}-e^{b_ix_{i-1}}]$\newline
We will draw $Z\sim Unif(0,1)$ and set $x = F^{-1}(z)$ for $i$th interval to get $X\sim F$,\newline
$x = F^{-1}(z) = \frac{1}{b_i} \{\displaystyle log[N_fb_iz + exp(a_i+b_ix_{i-1})] -a_i\}$

\subsubsection*{(c)}

\hspace{12 pt} The psuedo-code style software that would implement ARS is listed below.
%% begin.rcode PS52, eval=FALSE, echo=TRUE
%%# Problem 5.2: ARS
%%# [Reference] 
%%# http://www.amsta.leeds.ac.uk/~wally.gilks/adaptive.rejection/web_page/Welcome.html

%%# calculate the intercept and slope of the line set by succecutive points
%%lineCoef <- function(x,y){# x in ascending order
%%	return(data.frame(intercept=a,slope=b))
%%}

%%# calculate the intersection points above density curve
%%# given the input points on the density curve
%%intersec <- function(x,y){# x and y are the coordinates of the n points 
%%	# get line info for n-1 intersection points
%%	ab <- lineCoef(x=x,y=y)
%%	# extrapolate intersection points outside the curve
%%	qx[i] <- (a[i+1]-a[i-1])/(b[i-1]-b[i+1])
%%	qy[i] <- a[i-1] + b[i-1]* qx[i]
%%	return(cbind(qx,qy))
%%}

%%# get the set of points that construct the envelop
%%envelop <- function(x,y){# x and y are input points on the curve
%%	Q <- intersec(x,y)# all the upper points
%%	T <- PQ[order(PQ)]# in order
%%	return(data.frame(T))
%%}

%%# function that initializes input to sampling algorithm;
%%# The inner and env element include the points and lines 
%%# info for the inner chords and envelopes respectively.
%%# The region element includes the line info for each region 
%%# that is needed to be calculated
%%init <- function(x,y){
%%  # order x and y in the increasing order of x
%%	inner <- list(pts=data.frame(x=x,y=y), line=lineCoef(x,y))
%%	env <- list(pts=envelop(x,y),line=lineCoef(xy.env[1],xy.env[2]))
%%  reg <- rbind(ab.inner,ab.env)
%%	return(list(inner=inner,env=env,region=ab.reg))
%%}

%%# function to compute the areas of the envelop
%%area <- function(pts,lb=NULL, rb=NULL){# init from initialization
%%  # evaluation regions and points
%%  reg <- pts$region
%%  x<- pts$env$pts$x
%%  # calculate the area under the envelope
%%	ar[i] <- (exp(a+b*x[i+1])- exp(a+b*x[i]))/b
%%  # normalize area and compute cumulatives
%%  sum.ar <- sum(ar)       # raw total area
%%  ar <- ar/sum.ar         # normalized - sum to 1
%%  cum.ar <- cumsum(ar)    # cummulative
%%  return(list(ar=ar,cum.ar=cum.ar,sum.ar=sum.ar)	)
%%}

%%# Draw sample from envelope
%%sample.env <- function(pts,lb=NULL,rb=NULL){
%%  # calculate the area under the envelope
%%  ar <- area(pts,lb=lb,rb=rb)
%%  reg <- pts$region
%%  x<- pts$env$pts$x
%%  # line coefficients for that region
%%  a <- reg[rn,1]
%%  b <- reg[rn,2] 
%%  # sample a point from the envelope using inversion method
%%	u <- runif(1)
%%  # invert cumulative df to get sample point
%%	sx <- (log(u*b*ar$sum.ar + exp(b*x[rn]+a)) - a) /b    
%%	return(sx)
%%}

%%# test whether the sample should be accepted or not
%%# and store the evaluation information of the target function
%%test.sample <- function(sx,pts,func){
%%  x<- pts$env$pts$x
%%	u <- runif(1)	
%%	if (u < exp(r[1]-r[2])) { # simple acceptance step
%%		accept <- 1
%%	} else { # evaluation and rejection step
%%		if ( u < exp(func(sx)-r[2])) {
%%			accept <- 1
%%		}
%%		if (!(sx %in% x)){ # updating step: update pts
%%				x <- c(x, sx)
%%				y <- c(y, sy)
%%        pts <- init(x,y)
%%		}
%%	}
%%	return(list(accept=accept,pts=pts,xeval=x))	
%%}

%%# main function to implement the ARS algorithm;
%%# n: number of samples wanted;
%%# func: the log density;
%%# xinit: initial values for the X's, at least 3;
%%# lb: left bound;
%%# rb: right bound;
%%# The output is a list of 4 elements;
%%# sample: the sample values;
%%# xeval: the new x's that cause the func to be evaluated;
%%# pts: the points and line info for the chords and envelop;
%%ars <- function(n,func,xinit,lb=NULL,rb=NULL){
%%	# initialize
%%	s <- c()
%%	yinit <- func(xinit)
%%	pts <- init(xinit,yinit)
%%	# sample
%%  while (length(s)<n){
%%		sx <- sample.env(pts,lb=lb,rb=rb)
%%		tsx <- test.sample(sx,pts,func)
%%		if (tsx$accept) s <- c(s,sx)
%%		pts <- tsx$pts
%%    xeval <- tsx$xeval
%%	}
%%	return(list(sample=s,xeval=xeval,pts=pts))
%%}

%% end.rcode


\newpage
\section*{Problem 3}
We will explore the need in importance sampling that the sampling density have heavier tails than
the density of interest. We will estimate $EX$ and $E(X^2)$ with respect to density $f$ in this problem.

\subsubsection*{(a)}
\hspace{12 pt} Suppose $f\sim N(0,1)$ with sampling density $g\sim t(df=3)$. We sample $m=10000$ points
to extract histograms of estimates and weights in order to get an idea whether $Var(\hat\mu)$ is large.

As you can see from the results shown below, the variance is not so large since there are not many extreme
weights in the samples.

%% begin.rcode PS53a, cache=TRUE, results="markup", dev='cairo_pdf', fig.height=50, fig.align="center"
## (a) we want to estimate EX and VX for a standrad normal drawn from a t with 3 degrees of freedom
m <- 10000 # number of samples for each estimator
set.seed(0)
x <- rt(m, df = 3)  # i.e sample from g being a t with df=3
f <- dnorm(x)  # density of x under f
g <- dt(x, df = 3)  # density of x under g
w <- f/g  # weights
mean(x*w) #EX
mean((x*w)^2) #VX

par(mfrow=c(1,2),cex= 10, cex.main= 1.2)
hist(w)
hist(x*w)

%% end.rcode

\subsubsection*{(b)}
\hspace{12 pt} Suppose $f\sim t(df=3)$ with sampling density $g\sim N(0,1)$. We sample $m=10000$ points
to extract histograms of estimates and weights in order to get an idea whether $Var(\hat\mu)$ is large.

As the sampling density $t(df=3)$ has less heavier tails than the density of interest $N(0,1)$, there are
many extreme weights that have a very strong influence on the estimation variance, as shown below.


%% begin.rcode PS53b, cache=TRUE, results="markup", dev='cairo_pdf', fig.height=50, fig.align="center"
## (b) we want to estimate EX and EX^2 for a t with v=3 drawn from a standard normal
m <- 10000 # number of samples for each estimator
set.seed(0)
x <- rnorm(m)  # i.e sample from g being a standard normal
f <- dt(x, df = 3)  # density of x under f
g <- dnorm(x)  # density of x under g
w <- f/g  # weights
mean(x*w) #EX
mean((x*w)^2) #VX

par(mfrow=c(1,2),cex= 10, cex.main= 1.2)
hist(log(w))
hist(log(x*w))

%% end.rcode


\newpage
\section*{Problem 4}

\hspace{12 pt} This is a truncated/censored regression problem. In a given i.i.d. sample, stochastically, $c$ of the $n$ original observations
will be censored (point $y_i$ will be denoted as NA in the simulated test sample if and only if $y_i > \tau$), with the known 
indices set $IND=\{i^{obs},i^{mis}\}$ where $j=i^{obs}_1,...,i^{obs}_{n-c}$ and $k=i^{mis}_1,...,i^{mis}_c$. 
Here, we assume a simple linear regression model, 
$Y\sim X\beta + \epsilon$ with $Y_i\sim N(\beta_0+\beta_1x_i,\sigma^2)$, i.e. define $\beta=(\beta_0,\beta_1), \mu_i=(1, x_i)^T\beta$

\subsubsection*{(a)}
\hspace{12 pt} EM algorithm is designed to estimate the parameter set, $\theta = (\beta_0,\beta_1,\sigma^2)$. Here, we take the complete data to be
the available data plus the actual values of the truncated observations, i.e. $Y=\{Y^{obs}, m^{(t)}\}$, of which 
$m_k^{(t)}, k=i^{mis}_1,...,i^{mis}_c$ are values in place of the initially NA censored 
observations and they are functions of $\theta^{(t)}$, constant in terms of the maximization step.  
So the expected complete log likelihood is \newline
$logN(y_i|\theta) = \displaystyle -log\sqrt{2\pi\sigma^2} - \frac{(y_i-\mu_i)^2}{2\sigma^2}$ \newline
$logf(Y|\theta) =\displaystyle \sum_{j=i^{obs}_1}^{i^{obs}_{n-c}} log[N(Y^{obs}_{j}|\theta)I(Y^{obs}_{j}\leq \tau)]
                             + \sum_{k=i^{mis}_1}^{i^{mis}_c} log[N(m_k|\theta)I(m_k > \tau)] $ \newline
$Q(\theta|\theta^{(t)}) = \displaystyle E[logf(Y|\theta)|Y^{obs},\theta^{(t)}]
                        = \sum_{j=i^{obs}_1}^{i^{obs}_{n-c}} logN(Y^{obs}_{j}|\theta) 
                        + \sum_{k=i^{mis}_1}^{i^{mis}_c} logN(m_k^{(t)}|\theta) $ \newline
Here, $m_k^{(t)}\sim f(Y_{mis})$, truncated normal distribution,$N(\mu^{(t)},(\sigma^2)^{(t)})I(Y_{mis} > \tau)$ \newline
\hspace{12 pt} Then, the maximization step would give us updates on both $\theta$ and $m$:\newline
$m_k^{(t+1)} = E(m_k^{(t)}) = E(Y|Y > \tau) = \mu^{(t)} + \sigma^{(t)}\rho(\tau^*)$,
where $\displaystyle \rho(\tau^*) = \frac{\phi(\tau^*)}{1-\Phi(\tau^*)}$ , $\tau^*=(\tau-\mu^{(t)})/\sigma^{(t)}$, 
$\phi(\cdot)$ is the standard normal PDF and $\Phi(\cdot)$ is the standard normal CDF. \newline
$\beta^{(t)} = (X^TX)^{-1}X^TY$, where $Y=(Y^{obs}, m^{(t)})$ with corresponding $IND$. \newline
$\displaystyle (\sigma^2)^{(t)} = \widehat{Var}(Y_{obs})+\widehat{Var}(m^{(t)})
                                = \frac{1}{N}[(Y_{obs}-X_{obs}\beta)^T(Y_{obs}-X_{obs}\beta) 
                                              + ]$







\newpage
\section*{Problem 5}















\newpage
\section*{Problem 6}
\hspace{12 pt} In the work that won the Netflix Prize done by C. Volinsky etc. at AT\&T labs, a variation
on the singular value decomposition (SVD) of the matrix factorization methodology is employed.

Given a matrix $R$ that represents $m$ users (individual Netflix members) and $n$ items (movies)
with entries of movie ratings $r_{ui}$, $R=\{r_{ui}\}_{1\leq u\leq m, 1\leq i\leq n}$,
the SVD factorization-based methodology computes the besk rank-$f$ approximation
$R^f=P_{m\times f}Q_{n\times f}^T$, where $f\leq m,n$.

Due to the fact that most entries of $R$ in this movie recommendation problem are unknown,
the SVD here is used to extend the given data by filling in the unknow ratings by estimation 
$r_{ui} \sim R_{ui}^f = p_u^T \cdot q_i$, where $p_u$ is the $u$-th row of $P$ corresponding to user $u$
and $q_i$ is the $i$-th row of $Q$ corresponding to item $i$.

An EM-based algorithm is employed to iteratively compute the $R$ matrix SVD by doing
least square minimization of $\|R-PQ^T\|_F$ when alternating the fixed point between
matrix $P$ and $Q$, i.e. \newline
$Q^T \leftarrow (P^TP)^{-1}P^TR$ \newline
$P \leftarrow RQ(Q^TQ)^{-1}$\newline
Shrinkage is integrated to alleviate the overfitting problem and further reduce the estimation error,\newline
$Err(P,Q)\equiv \displaystyle\sum_{(u,i)}(r_{ui}-p_u^Tq_i)^2$.

To be more specific, assuming that we have already computed the first $f-1$ columns of
matrices $P,Q$. The pseudo code for computing the $f$-th column of matrices $P,Q$ is given below
\cite{KDD07}:

%% begin.rcode PS56, eval=FALSE, echo=TRUE
%%# Problem 5.6: EM based SVD variant
f <- f+1 # iteration
computeNext <- function(r,P,Q){ #known ratings, user factors, item factors
	# compute the f-th column of P,Q to fit given ratings
	# columns 1,...,f-1 were already computed
	a <- alpha
	e <- eps
	# compute residuals portion not explained by previous factors
	for (u in 1:m){
		for (i in 1:n){
			r[u,i] <- r[u,i] - crossprod(P[u,1:(f-1)],Q[i,1:(f-1)])
			n <- s[u,i] # support behind r
			r[u,i] <- n*r[u,i]/(n+a*f) #shrinkage
		}
	}
	# compute the f-th factor for each user and item by solving
	# many least square problems, each with a single unknown
	while (err(P,Q) < 1-e){
		for (u in 1:m){
			P[u,f] <- crossprod(r[u, ],Q[ ,f])/crossprod(Q[ ,f])
		}
		for (i in 1:n){
			Q[i,f] <- crossprod(r[ ,i],P[ ,f])/crossprod(P[ ,f])
		}
	}
	return(list(P=P,Q=Q))
}
%% end.rcode

At the end of the process, we obtain an approximation of all ratings in the form of a matrix product
$PQ^T$, with predictions for unrated movies. The eigenvectors from SVD are user and item factors, while
the eigenvalues are confidence indicators for the predictions.

\begin{thebibliography}{9}
\bibitem[1]{KDD07} R. Bell, Y. Koren, C. Volinsky, 
"Modeling relationships at multiple scales to improve accuracy of large recommender systems",
\textit{Proceedings of the 13th ACM SIGKDD International Conference on Knowledge Discovery and Data mining},
pp. 95-104, Aug. 2007, San Jose, CA.
\end{thebibliography}

\end{document}
