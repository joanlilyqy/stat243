\documentclass{article}

\usepackage{color}
\definecolor{dkgreen}{rgb}{0,0.6,0}
\definecolor{gray}{rgb}{0.5,0.5,0.5}
\definecolor{mauve}{rgb}{0.58,0,0.82}
\usepackage[margin=1in]{geometry}
\usepackage{fancyhdr}
\pagestyle{fancy}
\lhead{\today}
\chead{Ying Qiao SID:21412301}
\rhead{Stat243 Fa12: Problem Set 5}
\lfoot{}
\cfoot{\thepage}
\rfoot{}
\usepackage{graphicx}
\usepackage{textcomp}
\usepackage{lmodern}
\usepackage[T1]{fontenc}
\usepackage{listings}

%% for inline R code: if the inline code is not correctly parsed, you will see a message
\newcommand{\rinline}[1]{SOMETHING WORNG WITH knitr}
%% begin.rcode setup, include=FALSE
% opts_chunk$set(fig.path='figure/latex-', cache.path='cache/latex-', width=90)
%% end.rcode


\begin{document}
\section*{Problem 1}

\subsubsection*{(a)}

\hspace{12 pt} As expressed explictly in the paper, the purpose of the simulation is to assess the accuracy
of the proposed asymptotic approximation in finite samples and to examine the power of the EM test. 

The key metrics that the authors consider in the simulation study includes model order, null/alternative 
model selections, sample size choice and number of test repitions.

\subsubsection*{(b)}

\hspace{12 pt} The first choice is the model order, in this study, order of 2 and 3 is chosen to demonstrate the test statistics.
The next and probabaly the most important choices are the null/alternative model selections, i.e. the normal mixture parameter 
specifications ($\alpha, \theta, \sigma$ and different combinations). 
The final choices are the significance level (i.e. 1\% and 5\%), 
sample sizes (i.e. 200 and 400)and number of test repititions (i.e. 1000 and 5000) for the simulation study.

As for the data generating mechanism, the key aspects that will likely affect the statistical power of the test
should be the RNG of mixture normal models. Though not much details are given in the paper, the RNG of mixture
normal models are not that straightforward, and could have siginificant impact on the final results if naive
RNG yields a lot of overlaps and reductions.

\subsubsection*{(c)}

\hspace{12 pt} There are some data-generating scenarios that are missing in their simulation study.
First, for order-2 models, the mixture model parameter specifications have missed out $(\theta_1,\theta_2)$ with
both positive or negative numbers. The current 3 levels are all pairs of positive/negative numbers.
Similar scenarios exist with order-3 models.

Second, the alternative models are designed intentionally to mimic those null models, which though close to reality,
are missing some non-ideal scenarios of models with large discrepency.


\subsubsection*{(d)}

\hspace{12 pt} In order to test the hypothesis e.g. $H_0: m_0=2$ vs. $H_A: m_0 > 2$, the design space is quite large. 
We could use the principles of basic experiemental design to set up the simulation study, but it would be 
difficult to cover the whole space. As shown in the current setup, null models are just really a few 
samples from the huge design space, only 2 $(\alpha_1, \alpha_2)$ levels, 3 $(\theta_1, \theta_2)$ levels and
2 $(\sigma_1, \sigma_2)$ levels are considered to compose a full factorial design of $2\times 3\times 2=12$ levels.
With the continuous and wide selections of $\alpha, \theta, \sigma$ numbers, level choices that cover wide design
space will yield huge number of full-factorial design levels.

However, it is possible to increase the random sample size levels from 2 (i.e. 200,400) to more. And this will give
a much clearer trend of EM test power scaling with sample sizes.


\subsubsection*{(e)}

\hspace{12 pt} The tables have done a fairly good job in representating EM test power under different models and sample sizes.
However, there are inconsistencies between Table 4 and 6, as $\theta$ values split in Table 4 while $\alpha$ values
split in Table 6. There is not much explanation in the text and cause some confusion on the reader's side.

As for the figures, the representation could be better if the box plots are composed together with different labeling
and grouping, so that the readers can have more comparisons among different combinations.

At last, from the original text in the paper, there is no discussion on the issue of simulation uncertainty or standard
errors. It is not so convincing to the readers that they have done enough simulation replications. The choice on the
number of test repititions is rather arbitrary and lacks supporting evidence.


\subsubsection*{(f)}

\hspace{12 pt} From the interpretation of the results shown in Tables 4 and 6, we can conclude that the EM test power is positively 
correlated with the random sample size, as all the numbers in the $n=400$ column are larger than those in the $n=200$
column. This shows that the data generating mechanism does have large impact on the final results and should be 
analyzed more in the discussion and representation of the results.


\subsubsection*{(g)}

\hspace{12 pt} The JASA guidelines on simulation studies:
"Results Based on Computation - Papers reporting results based on computation should provide
enough information so that readers can evaluate the quality of the results. Such information
includes estimated accuracy of results, as well as descriptions of pseudorandom-number generators,
numerical algorithms, computers, programming languages, and major software components
that were used."

The authors fail to provide enough info suggested in the JASA guidelines on the simulation study in the paper. 
We do not see descriptions of pseudorandom-number generators and computers performance specifications that were used.

\end{document}
