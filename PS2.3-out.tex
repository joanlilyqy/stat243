\documentclass{article}\usepackage{graphicx, color}
%% maxwidth is the original width if it is less than linewidth
%% otherwise use linewidth (to make sure the graphics do not exceed the margin)
\makeatletter
\def\maxwidth{ %
  \ifdim\Gin@nat@width>\linewidth
    \linewidth
  \else
    \Gin@nat@width
  \fi
}
\makeatother

\IfFileExists{upquote.sty}{\usepackage{upquote}}{}
\definecolor{fgcolor}{rgb}{0.2, 0.2, 0.2}
\newcommand{\hlnumber}[1]{\textcolor[rgb]{0,0,0}{#1}}%
\newcommand{\hlfunctioncall}[1]{\textcolor[rgb]{0.501960784313725,0,0.329411764705882}{\textbf{#1}}}%
\newcommand{\hlstring}[1]{\textcolor[rgb]{0.6,0.6,1}{#1}}%
\newcommand{\hlkeyword}[1]{\textcolor[rgb]{0,0,0}{\textbf{#1}}}%
\newcommand{\hlargument}[1]{\textcolor[rgb]{0.690196078431373,0.250980392156863,0.0196078431372549}{#1}}%
\newcommand{\hlcomment}[1]{\textcolor[rgb]{0.180392156862745,0.6,0.341176470588235}{#1}}%
\newcommand{\hlroxygencomment}[1]{\textcolor[rgb]{0.43921568627451,0.47843137254902,0.701960784313725}{#1}}%
\newcommand{\hlformalargs}[1]{\textcolor[rgb]{0.690196078431373,0.250980392156863,0.0196078431372549}{#1}}%
\newcommand{\hleqformalargs}[1]{\textcolor[rgb]{0.690196078431373,0.250980392156863,0.0196078431372549}{#1}}%
\newcommand{\hlassignement}[1]{\textcolor[rgb]{0,0,0}{\textbf{#1}}}%
\newcommand{\hlpackage}[1]{\textcolor[rgb]{0.588235294117647,0.709803921568627,0.145098039215686}{#1}}%
\newcommand{\hlslot}[1]{\textit{#1}}%
\newcommand{\hlsymbol}[1]{\textcolor[rgb]{0,0,0}{#1}}%
\newcommand{\hlprompt}[1]{\textcolor[rgb]{0.2,0.2,0.2}{#1}}%

\usepackage{framed}
\makeatletter
\newenvironment{kframe}{%
 \def\at@end@of@kframe{}%
 \ifinner\ifhmode%
  \def\at@end@of@kframe{\end{minipage}}%
  \begin{minipage}{\columnwidth}%
 \fi\fi%
 \def\FrameCommand##1{\hskip\@totalleftmargin \hskip-\fboxsep
 \colorbox{shadecolor}{##1}\hskip-\fboxsep
     % There is no \\@totalrightmargin, so:
     \hskip-\linewidth \hskip-\@totalleftmargin \hskip\columnwidth}%
 \MakeFramed {\advance\hsize-\width
   \@totalleftmargin\z@ \linewidth\hsize
   \@setminipage}}%
 {\par\unskip\endMakeFramed%
 \at@end@of@kframe}
\makeatother

\definecolor{shadecolor}{rgb}{.97, .97, .97}
\definecolor{messagecolor}{rgb}{0, 0, 0}
\definecolor{warningcolor}{rgb}{1, 0, 1}
\definecolor{errorcolor}{rgb}{1, 0, 0}
\newenvironment{knitrout}{}{} % an empty environment to be redefined in TeX

\usepackage{alltt}

\usepackage{color}
\definecolor{dkgreen}{rgb}{0,0.6,0}
\definecolor{gray}{rgb}{0.5,0.5,0.5}
\definecolor{mauve}{rgb}{0.58,0,0.82}
\usepackage[margin=1in]{geometry}
\usepackage{fancyhdr}
\pagestyle{fancy}
\lhead{\today}
\chead{Ying Qiao SID:21412301}
\rhead{Stat243 Fa12: Problem Set 2}
\lfoot{}
\cfoot{\thepage}
\rfoot{}
\usepackage{graphicx}
\usepackage{textcomp}
\usepackage{lmodern}
\usepackage[T1]{fontenc}
\usepackage{listings}

%% for inline R code: if the inline code is not correctly parsed, you will see a message
\newcommand{\rinline}[1]{SOMETHING WORNG WITH knitr}




\begin{document}


\section*{Problem 3}

The American Presidency Project at UCSB has the text from all of the State of the Union
speeches by US presidents, in which the president speaks to Congress to report on the 
situation in the country. We will use web scraping, text formatting and pattern matching 
to grab the data; and then do some statistical analysis on them.

\subsection*{(a)}

From the website, I download the \textit{index.html} file and use pattern matching
to pull out the individual URLs for each speech in order to download individual HTML files.
Files are converted to UNIX line-ending using \textit{fromdos}.

\begin{knitrout}
\definecolor{shadecolor}{rgb}{0.969, 0.969, 0.969}\color{fgcolor}\begin{kframe}
\begin{alltt}
 \hlcomment{#### Download all html files}
 \hlfunctioncall{system}("wget -q -O \hlstring{'index_pres.html'} 
         \hlstring{'http://www.presidency.ucsb.edu/sou.php#axzz265cEKp1a'}")
 \hlfunctioncall{system}(\hlstring{"fromdos index_pres.html"})
 indexPres <- \hlfunctioncall{readLines}(\hlstring{'index_pres.html'},warn= FALSE)
 \hlcomment{# Get speech text source}
 patUrl1 <- 
 \hlstring{'\textbackslash{}\textbackslash{}s\{16\}<td width=\textbackslash{}\textbackslash{}\hlstring{"\textbackslash{}\textbackslash{}d\{2\}\textbackslash{}\textbackslash{}"} align=\textbackslash{}\textbackslash{}\hlstring{"center\textbackslash{}\textbackslash{}"} class=\textbackslash{}\textbackslash{}\hlstring{"doclist\textbackslash{}\textbackslash{}"}><a href=\textbackslash{}\textbackslash{}"'}
 indexPres <- indexPres[\hlfunctioncall{grep}(patUrl1,indexPres,perl= TRUE)]
 indexPres <- \hlfunctioncall{sapply}(indexPres, 
                     \hlfunctioncall{function}(x)\{\hlfunctioncall{gsub}(patUrl1,\hlstring{""},x)\},USE.NAMES= FALSE)
 patUrl2 <- \hlstring{'\textbackslash{}\textbackslash{}">\textbackslash{}\textbackslash{}d\{4\}<\textbackslash{}\textbackslash{}/a>(\textbackslash{}\textbackslash{}*|)<\textbackslash{}\textbackslash{}/td>'}
 indexPres <- \hlfunctioncall{sapply}(indexPres, 
                     \hlfunctioncall{function}(x)\{\hlfunctioncall{gsub}(patUrl2,\hlstring{""},x)\},USE.NAMES= FALSE)
 \hlcomment{# Get file id from the source url}
 patUrl3 <- 
 \hlstring{'http:\textbackslash{}\textbackslash{}/\textbackslash{}\textbackslash{}/www\textbackslash{}\textbackslash{}.presidency\textbackslash{}\textbackslash{}.ucsb\textbackslash{}\textbackslash{}.edu\textbackslash{}\textbackslash{}/ws\textbackslash{}\textbackslash{}/index\textbackslash{}\textbackslash{}.php\textbackslash{}\textbackslash{}?pid='}
 fileid <- \hlfunctioncall{sapply}(indexPres, 
                  \hlfunctioncall{function}(x)\{\hlfunctioncall{gsub}(patUrl3,\hlstring{""},x)\},USE.NAMES= FALSE)
 \hlcomment{# Download all files and convert to unix}
 \hlfunctioncall{sapply}(1:\hlfunctioncall{length}(fileid),
        \hlfunctioncall{function}(i)\{
        \hlfunctioncall{system}(\hlfunctioncall{paste}(\hlstring{"wget -q -O \hlstring{'"},fileid[i],\hlstring{".html'} \hlstring{'"},indexPres[i],\hlstring{"'}"},sep=\hlstring{""}));
        \hlfunctioncall{system}(\hlfunctioncall{paste}(\hlstring{"fromdos "},fileid[i],\hlstring{".html"},sep=\hlstring{""}))\})
\end{alltt}
\end{kframe}
\end{knitrout}



\subsection*{(b)}

For each speech, I use pattern matching to extract the body of the speech while retaining 
the name of the president and the year of the speech. The function is applied with vector
operations using \textit{sapply()}.

For the \textit{speechVec}, text pre-processing is done by 
\begin{enumerate}
\item
Replacing HTML line-end with UNIX ones 
\item
Removing all the HTML format operators
\item
Modifying all the HTML special characters to similar UTF-8 ones
\end{enumerate}

\newpage
\begin{knitrout}
\definecolor{shadecolor}{rgb}{0.969, 0.969, 0.969}\color{fgcolor}\begin{kframe}
\begin{alltt}
\hlcomment{# Import all *.html lines}
ff <- \hlfunctioncall{sapply}(fileid, \hlfunctioncall{function}(x) \{
    \hlfunctioncall{readLines}(\hlfunctioncall{paste}(x, \hlstring{".html"}, sep = \hlstring{""}), warn = FALSE)
\})
\hlcomment{# Get the president name}
patName <- \hlstring{"^<title>(.*?)<\textbackslash{}\textbackslash{}/title>"}
namePres <- ff[\hlfunctioncall{grep}(patName, ff, perl = TRUE)]
namePres <- \hlfunctioncall{sapply}(namePres, \hlfunctioncall{function}(x) \{
    \hlfunctioncall{gsub}(patName, \hlstring{"\textbackslash{}\textbackslash{}1"}, x)
\}, USE.NAMES = FALSE)
namePres <- \hlfunctioncall{sapply}(namePres, \hlfunctioncall{function}(x) \{
    \hlfunctioncall{return}(\hlfunctioncall{unlist}(\hlfunctioncall{strsplit}(x, \hlstring{":"}))[1])
\}, USE.NAMES = FALSE)
\hlcomment{# Get the talk date}
patDate <- \hlstring{"^.*<span class=\textbackslash{}\textbackslash{}\textbackslash{}"}docdate\textbackslash{}\textbackslash{}\textbackslash{}\hlstring{">(.*?)<\textbackslash{}\textbackslash{}/span>.*$"}
dateTalk <- ff[\hlfunctioncall{grep}(patDate, ff, perl = TRUE)]
dateTalk <- \hlfunctioncall{sapply}(dateTalk, \hlfunctioncall{function}(x) \{
    \hlfunctioncall{gsub}(patDate, \hlstring{"\textbackslash{}\textbackslash{}1"}, x)
\}, USE.NAMES = FALSE)
dateTalk <- \hlfunctioncall{sapply}(dateTalk, \hlfunctioncall{function}(x) \{
    \hlfunctioncall{gsub}(\hlstring{"^.*\textbackslash{}\textbackslash{}s"}, \hlstring{""}, x)
\}, USE.NAMES = FALSE)
\hlcomment{# Get the speech content text and prune for nice-format print}
patText <- \hlstring{"^.*<span class=\textbackslash{}\textbackslash{}\textbackslash{}"}displaytext\textbackslash{}\textbackslash{}\textbackslash{}\hlstring{">(.*?)<\textbackslash{}\textbackslash{}/span>.*$"}
speechVec <- ff[\hlfunctioncall{grep}(patText, ff, perl = TRUE)]
speechVec <- \hlfunctioncall{sapply}(speechVec, \hlfunctioncall{function}(x) \{
    \hlfunctioncall{gsub}(patText, \hlstring{"\textbackslash{}\textbackslash{}1<p>"}, x)
\}, USE.NAMES = FALSE)  \hlcomment{# grab speech text}
speechVec <- \hlfunctioncall{sapply}(speechVec, \hlfunctioncall{function}(x) \{
    \hlfunctioncall{gsub}(\hlstring{"(<p.*?>|<\textbackslash{}\textbackslash{}/p>|<br>)"}, \hlstring{"\textbackslash{}n"}, x)
\}, USE.NAMES = FALSE)  \hlcomment{# for line ending}
speechVec <- \hlfunctioncall{sapply}(speechVec, \hlfunctioncall{function}(x) \{
    \hlfunctioncall{gsub}(\hlstring{"<.*?>"}, \hlstring{""}, x)
\}, USE.NAMES = FALSE)  \hlcomment{# remove all html format}
speechVec <- \hlfunctioncall{sapply}(speechVec, \hlfunctioncall{function}(x) \{
    x <- \hlfunctioncall{gsub}(\hlstring{"&mdash;"}, \hlstring{" -- "}, x)
    x <- \hlfunctioncall{gsub}(\hlstring{"&nbsp;"}, \hlstring{"  "}, x)
    x <- \hlfunctioncall{gsub}(\hlstring{"&lsquo;"}, \hlstring{" ' "}, x)
    x <- \hlfunctioncall{gsub}(\hlstring{"&#8226;"}, \hlstring{" \textbackslash{}\textbackslash{}. "}, x)
    x <- \hlfunctioncall{gsub}(\hlstring{"&lt;"}, \hlstring{" < "}, x)
    x <- \hlfunctioncall{gsub}(\hlstring{"&deg;"}, \hlstring{" degree "}, x)
    x <- \hlfunctioncall{gsub}(\hlstring{"&pound;"}, \hlstring{" pound "}, x)
    x <- \hlfunctioncall{gsub}(\hlstring{"&fra.*?;"}, \hlstring{" 1/2 "}, x)
    x <- \hlfunctioncall{gsub}(\hlstring{"&O.*?;"}, \hlstring{"O"}, x)
    x <- \hlfunctioncall{gsub}(\hlstring{"&e.*?;"}, \hlstring{"e"}, x)
\}, USE.NAMES = FALSE)  \hlcomment{# html special char}
\end{alltt}
\end{kframe}
\end{knitrout}



\subsection*{(c)}

Each speech is stored as a single character vector with all non-text stripped out. The encoding is converted from
WINDOWS-1251 to UTF-8.
Meanwhile, the information about the tags of "Laughter" and "Applause" and the number of times it was used
are kept as a record for each speech. The \textit{speechVec[i]} will be printed out in a nicely-formatted manner. 

\begin{knitrout}
\definecolor{shadecolor}{rgb}{0.969, 0.969, 0.969}\color{fgcolor}\begin{kframe}
\begin{alltt}
\hlcomment{# Remove audience response \hlfunctioncall{tags} (laughter & applause)}
patLau <- \hlstring{"\textbackslash{}\textbackslash{}[.*?(Laughter|laughter).*?\textbackslash{}\textbackslash{}]"}
patApp <- \hlstring{"\textbackslash{}\textbackslash{}[.*?(Applause|applause).*?\textbackslash{}\textbackslash{}]"}
getlauNum <- \hlfunctioncall{function}(x) \{
    \hlfunctioncall{if} (\hlfunctioncall{length}(\hlfunctioncall{gregexpr}(patLau, x, perl = TRUE)[[1]]) == 1 && \hlfunctioncall{gregexpr}(patLau, 
        x, perl = TRUE)[[1]] == -1) \{
        \hlfunctioncall{return}(0)
    \} else \{
        \hlfunctioncall{return}(\hlfunctioncall{length}(\hlfunctioncall{gregexpr}(patLau, x, perl = TRUE)[[1]]))
    \}
\}
getappNum <- \hlfunctioncall{function}(x) \{
    \hlfunctioncall{if} (\hlfunctioncall{length}(\hlfunctioncall{gregexpr}(patApp, x, perl = TRUE)[[1]]) == 1 && \hlfunctioncall{gregexpr}(patApp, 
        x, perl = TRUE)[[1]] == -1) \{
        \hlfunctioncall{return}(0)
    \} else \{
        \hlfunctioncall{return}(\hlfunctioncall{length}(\hlfunctioncall{gregexpr}(patApp, x, perl = TRUE)[[1]]))
    \}
\}
lauNum <- \hlfunctioncall{sapply}(speechVec, getlauNum, USE.NAMES = FALSE)
appNum <- \hlfunctioncall{sapply}(speechVec, getappNum, USE.NAMES = FALSE)
speechVec <- \hlfunctioncall{sapply}(speechVec, \hlfunctioncall{function}(x) \{
    \hlfunctioncall{iconv}(x, from = \hlstring{"WINDOWS-1251"}, to = \hlstring{"UTF-8"}, sub = \hlstring{" "})
\})
speechVec <- \hlfunctioncall{sapply}(speechVec, \hlfunctioncall{function}(x) \{
    x <- \hlfunctioncall{gsub}(patLau, \hlstring{""}, x, perl = TRUE)
    x <- \hlfunctioncall{gsub}(patApp, \hlstring{""}, x, perl = TRUE)
\})
\hlfunctioncall{names}(speechVec) <- NULL
\end{alltt}
\end{kframe}
\end{knitrout}



\subsection*{(d)}

The collection of speeches is stored in a clean fashion of list elements. This is easy later for plotting
variables changes over time.

\begin{knitrout}
\definecolor{shadecolor}{rgb}{0.969, 0.969, 0.969}\color{fgcolor}\begin{kframe}
\begin{alltt}
listSpeech <- \hlfunctioncall{list}()
listSpeech$id <- fileid
listSpeech$name <- namePres
listSpeech$date <- \hlfunctioncall{as.integer}(dateTalk)
listSpeech$numLaughter <- lauNum
listSpeech$numApplause <- appNum
listSpeech$speech <- speechVec
\end{alltt}
\end{kframe}
\end{knitrout}



\subsection*{(e) (f)}

Words and sentences are extracted from each speech, and are stored as individual elements of a (rather long)
character vector. Counts are also done on both words and sentences.

\newpage
\begin{knitrout}
\definecolor{shadecolor}{rgb}{0.969, 0.969, 0.969}\color{fgcolor}\begin{kframe}
\begin{alltt}
\hlcomment{# Speech analysis}
getWords <- \hlfunctioncall{function}(x) \{
    x <- \hlfunctioncall{gsub}(\hlstring{"'"}, \hlstring{""}, x, perl = TRUE)
    x <- \hlfunctioncall{gsub}(\hlstring{"\textbackslash{}\textbackslash{}W+"}, \hlstring{" "}, x, perl = TRUE)
    xs <- \hlfunctioncall{unlist}(\hlfunctioncall{strsplit}(x, \hlstring{"[ ]+"}, perl = TRUE))
    \hlfunctioncall{return}(xs[xs != \hlstring{""}])
\}
getSents <- \hlfunctioncall{function}(x) \{
    x <- \hlfunctioncall{gsub}(\hlstring{" (Mr|Ms|Mrs|Dr|St|Sr|Jr)\textbackslash{}\textbackslash{}."}, \hlstring{"\textbackslash{}\textbackslash{}1"}, x, perl = TRUE)
    x <- \hlfunctioncall{gsub}(\hlstring{"[\textbackslash{}\textbackslash{}.!\textbackslash{}\textbackslash{}?][ \textbackslash{}t]+"}, \hlstring{"\textbackslash{}n"}, x, perl = TRUE)
    xs <- \hlfunctioncall{unlist}(\hlfunctioncall{strsplit}(x, \hlstring{"\textbackslash{}n"}, perl = TRUE))
    \hlfunctioncall{return}(xs[xs != \hlstring{""}])
\}
listSpeech$wc <- \hlfunctioncall{sapply}(speechVec, \hlfunctioncall{function}(x) \{
    \hlfunctioncall{return}(\hlfunctioncall{length}(\hlfunctioncall{getWords}(x)))
\}, USE.NAMES = FALSE)
listSpeech$sc <- \hlfunctioncall{sapply}(speechVec, \hlfunctioncall{function}(x) \{
    \hlfunctioncall{return}(\hlfunctioncall{length}(\hlfunctioncall{getSents}(x)))
\}, USE.NAMES = FALSE)
listSpeech$wMean <- \hlfunctioncall{sapply}(speechVec, \hlfunctioncall{function}(x) \{
    \hlfunctioncall{return}(\hlfunctioncall{mean}(\hlfunctioncall{nchar}(\hlfunctioncall{getWords}(x))))
\}, USE.NAMES = FALSE)
listSpeech$wSD <- \hlfunctioncall{sapply}(speechVec, \hlfunctioncall{function}(x) \{
    \hlfunctioncall{return}(\hlfunctioncall{sd}(\hlfunctioncall{nchar}(\hlfunctioncall{getWords}(x))))
\}, USE.NAMES = FALSE)
listSpeech$sMean <- \hlfunctioncall{sapply}(speechVec, \hlfunctioncall{function}(x) \{
    \hlfunctioncall{return}(\hlfunctioncall{mean}(\hlfunctioncall{nchar}(\hlfunctioncall{getSents}(x))))
\}, USE.NAMES = FALSE)
listSpeech$sSD <- \hlfunctioncall{sapply}(speechVec, \hlfunctioncall{function}(x) \{
    \hlfunctioncall{return}(\hlfunctioncall{sd}(\hlfunctioncall{nchar}(\hlfunctioncall{getSents}(x))))
\}, USE.NAMES = FALSE)
\end{alltt}
\end{kframe}
\end{knitrout}




\subsection*{(g) (h)}

We now start to extract some features of interest from the speeches to analyze how the speeches have
changed over time. The result of all this is a list with each element containing the information
about a speech: the speech as a single string, the vector of sentences, the vector of words, the
word counts, and the additional quantification of variables about the speech from (g) as well as
the non-verbal variables from (c).


\begin{enumerate}
\item
Length in words and sentences \textit{wc,sc}
\item
Average and SD of word and sentence lengths \textit{wMean,wSD,sMean,sSD}
\item
Number of quotations in each speech, mean length (in words), and SD of length (in words) of the 
quotations in each speech \textit{quoNum,quoMean,quoSD}
\item
The most common meaningful words, where non-meaningful words are pre-defined \textit{cmw}
\item
Counts of the following words or word stems: 
\begin{description}
\item[I, we]
\item[America{,n}]
\item[democra{cy,tic}]
\item[republic]
\item[Democrat{,ic}]
\item[Republican]
\item[free{,dom}]
\item[war]
\item[God] -- not including God bless
\item[God Bless]
\item[{Jesus, Christ, Christian}]
\item[Woman] -- I think would be interesting
\end{description}
\end{enumerate}


\begin{knitrout}
\definecolor{shadecolor}{rgb}{0.969, 0.969, 0.969}\color{fgcolor}\begin{kframe}
\begin{alltt}
tmpList <- \hlfunctioncall{matrix}(\hlfunctioncall{rep}(0, 226 * 15), nrow = 226, ncol = 15)
\hlcomment{## Speech list with element-wise analysis}
speechList <- \hlfunctioncall{list}()  \hlcomment{#empty list}
\hlfunctioncall{system}(\hlstring{"wget -O \hlstring{'common_words.txt'} \hlstring{'http://www.textfixer.com/resources/common-english-words.txt'}"})
commonWords <- \hlfunctioncall{readLines}(\hlstring{"common_words.txt"}, warn = FALSE)
commonWords <- \hlfunctioncall{unlist}(\hlfunctioncall{strsplit}(commonWords, \hlstring{","}, perl = TRUE))
\hlfunctioncall{for} (i in 1:\hlfunctioncall{length}(fileid)) \{
    ss <- \hlfunctioncall{list}()
    \hlcomment{# Global attr}
    ss$id <- fileid[i]
    ss$name <- namePres[i]
    ss$date <- \hlfunctioncall{as.integer}(dateTalk[i])
    ss$numLaughter <- lauNum[i]
    ss$numApplause <- appNum[i]
    ss$speech <- speechVec[i]
    \hlcomment{# Indiv attr}
    talkWords <- \hlfunctioncall{getWords}(speechVec[i])
    talkSents <- \hlfunctioncall{getSents}(speechVec[i])
    ss$words <- talkWords  \hlcomment{# words vector}
    ss$sents <- talkSents  \hlcomment{# sentence vector}
    ss$wc <- \hlfunctioncall{length}(talkWords)  \hlcomment{# word count}
    ss$sc <- \hlfunctioncall{length}(talkSents)  \hlcomment{# sentence count}
    ss$wMean <- \hlfunctioncall{mean}(\hlfunctioncall{nchar}(talkWords))  \hlcomment{# avg word length}
    ss$wSD <- \hlfunctioncall{sd}(\hlfunctioncall{nchar}(talkWords))  \hlcomment{# word length sd}
    ss$sMean <- \hlfunctioncall{mean}(\hlfunctioncall{nchar}(talkSents))  \hlcomment{# avg sentence length}
    ss$sSD <- \hlfunctioncall{sd}(\hlfunctioncall{nchar}(talkSents))  \hlcomment{# sentence length sd; ss[14]}
    patQuo <- \hlstring{"\textbackslash{}"}(.*?)\textbackslash{}\hlstring{""}  # quotation pattern
    quo <- talkSents[\hlfunctioncall{grep}(patQuo, talkSents, perl = TRUE)]
    \hlfunctioncall{if} (\hlfunctioncall{length}(quo) != 0) \{
        \hlcomment{# get quotation attr}
        quo <- \hlfunctioncall{sapply}(quo, \hlfunctioncall{function}(x) \{
            \hlfunctioncall{gsub}(patQuo, \hlstring{"\textbackslash{}\textbackslash{}1"}, x)
        \}, USE.NAMES = FALSE)
        ss$quoNum <- \hlfunctioncall{length}(quo)
        ss$quoMean <- \hlfunctioncall{mean}(\hlfunctioncall{nchar}(quo))
        ss$quoSD <- \hlfunctioncall{sd}(\hlfunctioncall{nchar}(quo))
    \} else \{
        ss$quoNum <- 0
        ss$quoMean <- 0
        ss$quoSD <- 0
    \}  \hlcomment{#ss[17]}
    
    cmw <- \hlfunctioncall{sort}(\hlfunctioncall{table}(talkWords), decreasing = TRUE)
    cmw <- cmw[\hlfunctioncall{which}(!(\hlfunctioncall{names}(cmw) %in% commonWords))]  \hlcomment{# get meaningful words}
    ss$cmw <- cmw[cmw >= 10]  \hlcomment{#arbitrary cut-off for display}
    ss$strIwe <- cmw[\hlfunctioncall{grep}(\hlstring{"^(I|[Ww]e)$"}, \hlfunctioncall{names}(cmw))]  #string \hlstring{'I|We'}; ss[19]
    ss$strAme <- cmw[\hlfunctioncall{grep}(\hlstring{"[Aa]\hlfunctioncall{merica}(|n)"}, \hlfunctioncall{names}(cmw))]
    ss$strDem <- cmw[\hlfunctioncall{grep}(\hlstring{"[Dd]\hlfunctioncall{emocra}(cy|tic)"}, \hlfunctioncall{names}(cmw))]
    ss$strRep <- cmw[\hlfunctioncall{grep}(\hlstring{"[Rr]\hlfunctioncall{epublic}(|n)"}, \hlfunctioncall{names}(cmw))]
    ss$strFree <- cmw[\hlfunctioncall{grep}(\hlstring{"^[Ff]\hlfunctioncall{ree}(|dom)$"}, \hlfunctioncall{names}(cmw))]
    ss$strWar <- cmw[\hlfunctioncall{grep}(\hlstring{"^[Ww]\hlfunctioncall{ar}(|s)$"}, \hlfunctioncall{names}(cmw))]
    ss$strGod <- cmw[\hlfunctioncall{grep}(\hlstring{"^[Gg]\hlfunctioncall{od}(|s)$"}, \hlfunctioncall{names}(cmw))]
    ss$strChr <- cmw[\hlfunctioncall{grep}(\hlstring{"(Jesus|Christ|Christian)"}, \hlfunctioncall{names}(cmw))]
    ss$strWoman <- cmw[\hlfunctioncall{grep}(\hlstring{"^[Ww]om[ae]n$"}, \hlfunctioncall{names}(cmw))]  #Mystring \hlstring{'Woman'}; ss[27]
    ssGodBless <- talkSents[\hlfunctioncall{grep}(\hlstring{"[Gg]od [Bb]less"}, talkSents, perl = TRUE)]
    \hlfunctioncall{if} (\hlfunctioncall{length}(ssGodBless) != 0) \{
        \hlcomment{#string \hlstring{'God Bless'} from sentences}
        ss$strGodBless <- \hlfunctioncall{sapply}(ssGodBless, \hlfunctioncall{function}(x) \{
            \hlfunctioncall{return}(\hlfunctioncall{length}(\hlfunctioncall{gregexpr}(\hlstring{"[Gg]od [Bb]less"}, x, perl = TRUE)[[1]]))
        \}, USE.NAMES = FALSE)
    \} else \{
        ss$strGodBless <- 0
    \}
    \hlcomment{# add to speechList and listSpeech}
    speechList[[i]] <- ss
    tmpList[i, 1:3] <- \hlfunctioncall{unlist}(ss[15:17])  \hlcomment{#quo}
    tmpList[i, 4:13] <- \hlfunctioncall{sapply}(ss[19:28], sum)  \hlcomment{#cmw}
\}
\hlcomment{# prepare for plotting}
listSpeech$quoNum <- tmpList[, 1]
listSpeech$quoMean <- tmpList[, 2]
listSpeech$quoSD <- tmpList[, 3]
listSpeech$strIwe <- tmpList[, 4]
listSpeech$strAme <- tmpList[, 5]
listSpeech$strDem <- tmpList[, 6]
listSpeech$strRep <- tmpList[, 7]
listSpeech$strFree <- tmpList[, 8]
listSpeech$strWar <- tmpList[, 9]
listSpeech$strGod <- tmpList[, 10]
listSpeech$strChr <- tmpList[, 11]
listSpeech$strWoman <- tmpList[, 12]
listSpeech$strGodBless <- tmpList[, 13]
\end{alltt}
\end{kframe}
\end{knitrout}



\subsection*{(i) (j)}

Some basic plots that show how the variables have changed over time are given below.





\begin{knitrout}
\definecolor{shadecolor}{rgb}{0.969, 0.969, 0.969}\color{fgcolor}\begin{kframe}
\begin{alltt}
\hlcomment{# Create plots}
\hlfunctioncall{attach}(listSpeech)
plotTitle1 = \hlstring{"Presidential Speech Statistics Over Time"}
xlab = \hlstring{"Year"}
\hlcomment{# over time}
\hlfunctioncall{layout}(\hlfunctioncall{matrix}(\hlfunctioncall{c}(1, 1, 2, 3), 2, 2, byrow = TRUE))
\hlfunctioncall{par}(cex.main = 1)
typ = \hlstring{"h"}
col = \hlstring{"blue"}
lwd = 6
\hlfunctioncall{plot}(date, wc, type = typ, col = col, lwd = lwd, main = plotTitle1, xlab = xlab, 
    ylab = \hlstring{"Word Count"})
typ = \hlstring{"b"}
lwd = 2
col = \hlstring{"brown"}
\hlfunctioncall{plot}(date, wMean, type = typ, col = col, lwd = lwd, main = plotTitle1, xlab = xlab, 
    ylab = \hlstring{"Avg Word Length"})
\hlfunctioncall{plot}(date, wSD, type = typ, col = col, lwd = lwd, main = plotTitle1, xlab = xlab, 
    ylab = \hlstring{"Word Length Std Dev"})
\end{alltt}
\end{kframe}\includegraphics[width=.8\textwidth]{figure/latex-PS23-fig1} \begin{kframe}\begin{alltt}

\hlfunctioncall{layout}(\hlfunctioncall{matrix}(\hlfunctioncall{c}(1, 1, 2, 3), 2, 2, byrow = TRUE))
\hlfunctioncall{par}(cex.main = 1)
typ = \hlstring{"h"}
col = \hlstring{"blue"}
lwd = 6
\hlfunctioncall{plot}(date, sc, type = typ, col = col, lwd = lwd, main = plotTitle1, xlab = xlab, 
    ylab = \hlstring{"Sentence Count"})
typ = \hlstring{"b"}
lwd = 2
col = \hlstring{"brown"}
\hlfunctioncall{plot}(date, sMean, type = typ, col = col, lwd = lwd, main = plotTitle1, xlab = xlab, 
    ylab = \hlstring{"Avg Sentence Length"})
\hlfunctioncall{plot}(date, sSD, type = typ, col = col, lwd = lwd, main = plotTitle1, xlab = xlab, 
    ylab = \hlstring{"Sentence Length Std Dev"})
\end{alltt}
\end{kframe}\includegraphics[width=.8\textwidth]{figure/latex-PS23-fig2} \begin{kframe}\begin{alltt}

\hlfunctioncall{layout}(\hlfunctioncall{matrix}(\hlfunctioncall{c}(1, 1, 2, 3), 2, 2, byrow = TRUE))
\hlfunctioncall{par}(cex.main = 1)
typ = \hlstring{"h"}
lwd = 6
col = \hlstring{"blue"}
\hlfunctioncall{plot}(date, quoNum, type = typ, col = col, lwd = lwd, main = plotTitle1, xlab = xlab, 
    ylab = \hlstring{"Quotation Count"})
typ = \hlstring{"b"}
lwd = 2
col = \hlstring{"brown"}
\hlfunctioncall{plot}(date, quoMean, type = typ, col = col, lwd = lwd, main = plotTitle1, xlab = xlab, 
    ylab = \hlstring{"Avg Quotation Length"})
\hlfunctioncall{plot}(date, quoSD, type = typ, col = col, lwd = lwd, main = plotTitle1, xlab = xlab, 
    ylab = \hlstring{"Quotation Length Std Dev"})
\end{alltt}
\end{kframe}\includegraphics[width=.8\textwidth]{figure/latex-PS23-fig3} \begin{kframe}\begin{alltt}
\hlcomment{# recent years}
plotTitle2 = \hlstring{"Presidential Speech Statistics in Modern U.S.A."}

recY <- 1:\hlfunctioncall{max}(\hlfunctioncall{c}(\hlfunctioncall{which}(numLaughter > 0), \hlfunctioncall{which}(numApplause > 0)))
typ = \hlstring{"h"}
lwd = 6
col = \hlstring{"blue"}
\hlfunctioncall{par}(mfrow = \hlfunctioncall{c}(2, 1), cex.main = 1)
\hlfunctioncall{plot}(date[recY], numLaughter[recY], type = typ, col = col, lwd = lwd, main = plotTitle2, 
    xlab = xlab, ylab = \hlstring{"Laughter Count"})
\hlfunctioncall{plot}(date[recY], numApplause[recY], type = typ, col = col, lwd = lwd, main = plotTitle2, 
    xlab = xlab, ylab = \hlstring{"Applause Count"})
\end{alltt}
\end{kframe}\includegraphics[width=.8\textwidth]{figure/latex-PS23-fig4} \begin{kframe}\begin{alltt}
\hlcomment{# Rep vs Dem}
plotTitle3 = \hlstring{"Republican vs. Democratic Presidential Speech \hlfunctioncall{Statistics} (Since 1932)"}

repPres <- \hlfunctioncall{c}(\hlstring{"Dwight D. Eisenhower"}, \hlstring{"Richard Nixon"}, \hlstring{"Gerald R. Ford"}, \hlstring{"Ronald Reagan"}, 
    \hlstring{"George Bush"}, \hlstring{"George W. Bush"})
demPres <- \hlfunctioncall{c}(\hlstring{"Franklin D. Roosevelt"}, \hlstring{"Harry S. Truman"}, \hlstring{"John F. Kennedy"}, 
    \hlstring{"Lyndon B. Johnson"}, \hlstring{"Jimmy Carter"}, \hlstring{"William J. Clinton"}, \hlstring{"Barack Obama"})
repY <- \hlfunctioncall{which}(name %in% repPres)
demY <- \hlfunctioncall{which}(name %in% demPres)
\hlfunctioncall{layout}(\hlfunctioncall{matrix}(1:16, 8, 2))
\hlfunctioncall{par}(mar = \hlfunctioncall{c}(1.5, 6, 1.5, 6))
col = \hlstring{"blue"}
cex = 0.6
\hlfunctioncall{boxplot}(wc[repY], col = col, ylim = \hlfunctioncall{c}(2000, 10000))
\hlfunctioncall{legend}(\hlstring{"topleft"}, \hlstring{"Rep. Word Count"}, cex = cex)
\hlfunctioncall{boxplot}(sc[repY], col = col, ylim = \hlfunctioncall{c}(100, 800))
\hlfunctioncall{legend}(\hlstring{"topleft"}, \hlstring{"Rep. Sentence Count"}, cex = cex)
\hlfunctioncall{boxplot}(quoNum[repY], col = col, ylim = \hlfunctioncall{c}(0, 12))
\hlfunctioncall{legend}(\hlstring{"topleft"}, \hlstring{"Rep. Quotation Count"}, cex = cex)
\hlfunctioncall{boxplot}(numLaughter[repY], col = col, ylim = \hlfunctioncall{c}(0, 8))
\hlfunctioncall{legend}(\hlstring{"topleft"}, \hlstring{"Rep. Laughter Count"}, cex = cex)
\hlfunctioncall{boxplot}(numApplause[repY], col = col, ylim = \hlfunctioncall{c}(0, 4))
\hlfunctioncall{legend}(\hlstring{"topleft"}, \hlstring{"Rep. Applause Count"}, cex = cex)
\hlfunctioncall{boxplot}(wMean[repY], col = col)
\hlfunctioncall{legend}(\hlstring{"topleft"}, \hlstring{"Rep. Avg Word Length"}, cex = cex)
\hlfunctioncall{boxplot}(sMean[repY], col = col, ylim = \hlfunctioncall{c}(90, 150))
\hlfunctioncall{legend}(\hlstring{"topleft"}, \hlstring{"Rep. Avg Sentence Length"}, cex = cex)
\hlfunctioncall{boxplot}(quoMean[repY], col = col, ylim = \hlfunctioncall{c}(0, 400))
\hlfunctioncall{legend}(\hlstring{"topleft"}, \hlstring{"Rep. Avg Quotation Length"}, cex = cex)
col = \hlstring{"brown"}
cex = 0.6
\hlfunctioncall{boxplot}(wc[demY], col = col, ylim = \hlfunctioncall{c}(2000, 10000))
\hlfunctioncall{legend}(\hlstring{"topleft"}, \hlstring{"Dem. Word Count"}, cex = cex)
\hlfunctioncall{boxplot}(sc[demY], col = col, ylim = \hlfunctioncall{c}(100, 800))
\hlfunctioncall{legend}(\hlstring{"topleft"}, \hlstring{"Dem. Sentence Count"}, cex = cex)
\hlfunctioncall{boxplot}(quoNum[demY], col = col, ylim = \hlfunctioncall{c}(0, 12))
\hlfunctioncall{legend}(\hlstring{"topleft"}, \hlstring{"Dem. Quotation Count"}, cex = cex)
\hlfunctioncall{boxplot}(numLaughter[demY], col = col, ylim = \hlfunctioncall{c}(0, 8))
\hlfunctioncall{legend}(\hlstring{"topleft"}, \hlstring{"Dem. Laughter Count"}, cex = cex)
\hlfunctioncall{boxplot}(numApplause[demY], col = col, ylim = \hlfunctioncall{c}(0, 4))
\hlfunctioncall{legend}(\hlstring{"topleft"}, \hlstring{"Dem. Applause Count"}, cex = cex)
\hlfunctioncall{boxplot}(wMean[demY], col = col)
\hlfunctioncall{legend}(\hlstring{"topleft"}, \hlstring{"Dem. Avg Word Length"}, cex = cex)
\hlfunctioncall{boxplot}(sMean[demY], col = col, ylim = \hlfunctioncall{c}(90, 150))
\hlfunctioncall{legend}(\hlstring{"topleft"}, \hlstring{"Dem. Avg Sentence Length"}, cex = cex)
\hlfunctioncall{boxplot}(quoMean[demY], col = col, ylim = \hlfunctioncall{c}(0, 400))
\hlfunctioncall{legend}(\hlstring{"topleft"}, \hlstring{"Dem. Avg Quotation Length"}, cex = cex)
\end{alltt}
\end{kframe}\includegraphics[width=.8\textwidth]{figure/latex-PS23-fig5} \begin{kframe}\begin{alltt}
\hlcomment{# extra}

\hlfunctioncall{par}(mfrow = \hlfunctioncall{c}(6, 1), cex = 0.5, cex.main = 1)
typ = \hlstring{"h"}
col = \hlstring{"blue"}
lwd = 6
\hlfunctioncall{par}(mar = \hlfunctioncall{c}(0.5, 6, 4, 6))
\hlfunctioncall{plot}(date, strIwe, type = typ, col = col, lwd = lwd, main = plotTitle1, ylab = \hlstring{"\hlstring{'I'}\hlstring{'We'} Count"})
\hlfunctioncall{par}(mar = \hlfunctioncall{c}(0.5, 6, 0.5, 6))
\hlfunctioncall{plot}(date, strAme, type = typ, col = col, lwd = lwd, ylab = \hlstring{"\hlstring{'America'} Count"})
\hlfunctioncall{par}(mar = \hlfunctioncall{c}(0.5, 6, 0.5, 6))
\hlfunctioncall{plot}(date, strFree, type = typ, col = col, lwd = lwd, ylab = \hlstring{"\hlstring{'Free/dom'} Count"})
\hlfunctioncall{par}(mar = \hlfunctioncall{c}(0.5, 6, 0.5, 6))
\hlfunctioncall{plot}(date, strWar, type = typ, col = col, lwd = lwd, ylab = \hlstring{"\hlstring{'War'} Count"})
\hlfunctioncall{par}(mar = \hlfunctioncall{c}(0.5, 6, 0.5, 6))
\hlfunctioncall{plot}(date, strGod, type = typ, col = col, lwd = lwd, ylab = \hlstring{"\hlstring{'God'} Count"})
\hlfunctioncall{par}(mar = \hlfunctioncall{c}(4, 6, 0.5, 6))
\hlfunctioncall{plot}(date, strWoman, type = typ, col = col, lwd = lwd, xlab = xlab, ylab = \hlstring{"\hlstring{'Woman'} Count"})
\end{alltt}
\end{kframe}\includegraphics[width=.8\textwidth]{figure/latex-PS23-fig6} \begin{kframe}\begin{alltt}

\hlfunctioncall{par}(mfrow = \hlfunctioncall{c}(2, 2), mar = \hlfunctioncall{c}(1, 2, 1, 2))
cex = 1
col = \hlstring{"blue"}
\hlfunctioncall{boxplot}(strRep[repY], col = col, ylim = \hlfunctioncall{c}(0, 15))
\hlfunctioncall{legend}(\hlstring{"topleft"}, \hlstring{"Rep. \hlstring{'Rep'} Count"}, cex = cex)
col = \hlstring{"purple"}
\hlfunctioncall{boxplot}(strDem[repY], col = col, ylim = \hlfunctioncall{c}(0, 15))
\hlfunctioncall{legend}(\hlstring{"topleft"}, \hlstring{"Rep. \hlstring{'Dem'} Count"}, cex = cex)
\hlfunctioncall{boxplot}(strRep[demY], col = col, ylim = \hlfunctioncall{c}(0, 15))
\hlfunctioncall{legend}(\hlstring{"topleft"}, \hlstring{"Dem. \hlstring{'Rep'} Count"}, cex = cex)
col = \hlstring{"brown"}
\hlfunctioncall{boxplot}(strDem[demY], col = col, ylim = \hlfunctioncall{c}(0, 15))
\hlfunctioncall{legend}(\hlstring{"topleft"}, \hlstring{"Dem. \hlstring{'Dem'} Count"}, cex = cex)
\end{alltt}
\end{kframe}\includegraphics[width=.8\textwidth]{figure/latex-PS23-fig7} \begin{kframe}\begin{alltt}

getDigts <- \hlfunctioncall{function}(x) \{
    x <- \hlfunctioncall{gsub}(\hlstring{"(\textbackslash{}\textbackslash{}d)[,\textbackslash{}\textbackslash{}.](\textbackslash{}\textbackslash{}d)"}, \hlstring{"\textbackslash{}\textbackslash{}1\textbackslash{}\textbackslash{}2"}, x, perl = TRUE)
    x <- \hlfunctioncall{gsub}(\hlstring{"\textbackslash{}\textbackslash{}D"}, \hlstring{"\textbackslash{}n"}, x, perl = TRUE)
    xs <- \hlfunctioncall{unlist}(\hlfunctioncall{strsplit}(x, \hlstring{"\textbackslash{}n"}, perl = TRUE))
    \hlfunctioncall{return}(xs[xs != \hlstring{""}])
\}
listSpeech$dc <- \hlfunctioncall{sapply}(speechVec, \hlfunctioncall{function}(x) \{
    \hlfunctioncall{return}(\hlfunctioncall{length}(\hlfunctioncall{getDigts}(x)))
\}, USE.NAMES = FALSE)
typ = \hlstring{"h"}
col = \hlstring{"blue"}
lwd = 6
\hlfunctioncall{plot}(date, listSpeech$dc, type = typ, col = col, lwd = lwd, main = plotTitle1, 
    xlab = xlab, ylab = \hlstring{"Digits Count"})
\end{alltt}
\end{kframe}\includegraphics[width=.8\textwidth]{figure/latex-PS23-fig8} 
\end{knitrout}







And for presidents since Franklin Roosevelt in 1932, comparison between Republican presidents 
(Eisenhower, Nixon, Ford, Reagan, G. Bush, G.W. Bush) and Democratic presidents 
(Roosevelt, Truman, Kennedy, Johnson, Carter, Clinton, Obama) are also given.









Some additional research and/or additional thinking to come up with additional
variables that quantify speech in interesting ways. Do some plotting that illustrates how the
speeches have changed over time.



\end{document}
