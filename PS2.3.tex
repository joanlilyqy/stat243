\documentclass{article}

\usepackage{color}
\definecolor{dkgreen}{rgb}{0,0.6,0}
\definecolor{gray}{rgb}{0.5,0.5,0.5}
\definecolor{mauve}{rgb}{0.58,0,0.82}
\usepackage[margin=1in]{geometry}
\usepackage{fancyhdr}
\pagestyle{fancy}
\lhead{\today}
\chead{Ying Qiao SID:21412301}
\rhead{Stat243 Fa12: Problem Set 2}
\lfoot{}
\cfoot{\thepage}
\rfoot{}
\usepackage{graphicx}
\usepackage{textcomp}
\usepackage{lmodern}
\usepackage[T1]{fontenc}
\usepackage{listings}

%% for inline R code: if the inline code is not correctly parsed, you will see a message
\newcommand{\rinline}[1]{SOMETHING WORNG WITH knitr}
%% begin.rcode setup, include=FALSE
% opts_chunk$set(fig.path='figure/latex-', cache.path='cache/latex-')
%% end.rcode


\begin{document}


\section*{Problem 3}

The American Presidency Project at UCSB has the text from all of the State of the Union
speeches by US presidents, in which the president speaks to Congress to report on the 
situation in the country. We will use web scraping, text formatting and pattern matching 
to grab the data; and then do some statistical analysis on them.

\subsection*{(a)}

From the website, I download the \textit{index.html} file and use pattern matching
to pull out the individual URLs for each speech in order to download individual HTML files.
Files are converted to UNIX line-ending using \textit{fromdos}.

%% begin.rcode eval=FALSE, tidy=FALSE
%% #### Download all html files
%% system("wget -q -O 'index_pres.html' 
%%         'http://www.presidency.ucsb.edu/sou.php#axzz265cEKp1a'")
%% system("fromdos index_pres.html")
%% indexPres <- readLines('index_pres.html',warn= FALSE)
%% # Get speech text source
%% patUrl1 <- 
%% '\\s{16}<td width=\\"\\d{2}\\" align=\\"center\\" class=\\"doclist\\"><a href=\\"'
%% indexPres <- indexPres[grep(patUrl1,indexPres,perl= TRUE)]
%% indexPres <- sapply(indexPres, 
%%                     function(x){gsub(patUrl1,"",x)},USE.NAMES= FALSE)
%% patUrl2 <- '\\">\\d{4}<\\/a>(\\*|)<\\/td>'
%% indexPres <- sapply(indexPres, 
%%                     function(x){gsub(patUrl2,"",x)},USE.NAMES= FALSE)
%% # Get file id from the source url
%% patUrl3 <- 
%% 'http:\\/\\/www\\.presidency\\.ucsb\\.edu\\/ws\\/index\\.php\\?pid='
%% fileid <- sapply(indexPres, 
%%                  function(x){gsub(patUrl3,"",x)},USE.NAMES= FALSE)
%% # Download all files and convert to unix
%% sapply(1:length(fileid),
%%        function(i){
%%        system(paste("wget -q -O '",fileid[i],".html' '",indexPres[i],"'",sep=""));
%%        system(paste("fromdos ",fileid[i],".html",sep=""))})
%% end.rcode


\subsection*{(b)}

For each speech, I use pattern matching to extract the body of the speech while retaining 
the name of the president and the year of the speech. The function is applied with vector
operations using \textit{sapply()}.

For the \textit{speechVec}, text pre-processing is done by 
\begin{enumerate}
\item
Replacing HTML line-end with UNIX ones 
\item
Removing all the HTML format operators
\item
Modifying all the HTML special characters to similar UTF-8 ones
\end{enumerate}

\newpage
%% begin.rcode eval=FALSE
%% # Import all *.html lines
%%ff <- sapply(fileid, function(x){readLines(paste(x,".html",sep=""), warn= FALSE)})
%% # Get the president name
%%patName <- "^<title>(.*?)<\\/title>"
%%namePres <- ff[grep(patName,ff,perl= TRUE)]
%%namePres <- sapply(namePres, function(x){gsub(patName,"\\1",x)},USE.NAMES= FALSE)
%%namePres <- sapply(namePres, function(x){return(unlist(strsplit(x, ":"))[1])},USE.NAMES= FALSE)
%%# Get the talk date
%%patDate <- '^.*<span class=\\"docdate\\">(.*?)<\\/span>.*$'
%%dateTalk <- ff[grep(patDate,ff,perl= TRUE)]
%%dateTalk <- sapply(dateTalk, function(x){gsub(patDate,"\\1",x)},USE.NAMES= FALSE)
%%dateTalk <- sapply(dateTalk, function(x){gsub("^.*\\s","",x)},USE.NAMES= FALSE)
%%# Get the speech content text and prune for nice-format print
%%patText <- '^.*<span class=\\"displaytext\\">(.*?)<\\/span>.*$'
%%speechVec <- ff[grep(patText,ff,perl= TRUE)]
%%speechVec <- sapply(speechVec, function(x){gsub(patText,"\\1<p>",x)},USE.NAMES= FALSE) # grab speech text
%%speechVec <- sapply(speechVec, function(x){gsub("(<p.*?>|<\\/p>|<br>)","\n",x)},USE.NAMES= FALSE) # for line ending
%%speechVec <- sapply(speechVec, function(x){gsub("<.*?>","",x)},USE.NAMES= FALSE) # remove all html format
%%speechVec <- sapply(speechVec, function(x){x <- gsub("&mdash;"," -- ",x);x <- gsub("&nbsp;","  ",x);
%%                                           x <- gsub("&lsquo;"," ' ",x);     x <- gsub("&#8226;"," \\. ",x);
%%                                           x <- gsub("&lt;"," < ",x);        x <- gsub("&deg;"," degree ",x);
%%                                           x <- gsub("&pound;"," pound ",x); x <- gsub("&fra.*?;"," 1/2 ",x);
%%                                           x <- gsub("&O.*?;","O",x);        x <- gsub("&e.*?;","e",x);                                     
%%                                          },USE.NAMES= FALSE) # html special char
%% end.rcode


\subsection*{(c)}

Each speech is stored as a single character vector with all non-text stripped out. The encoding is converted from
WINDOWS-1251 to UTF-8.
Meanwhile, the information about the tags of "Laughter" and "Applause" and the number of times it was used
are kept as a record for each speech. The \textit{speechVec[i]} will be printed out in a nicely-formatted manner. 

%% begin.rcode eval=FALSE
%%# Remove audience response tags (laughter & applause)
%%patLau <- '\\[.*?(Laughter|laughter).*?\\]'
%%patApp <- '\\[.*?(Applause|applause).*?\\]'
%%getlauNum <- function(x) {
%%          if (length(gregexpr(patLau,x,perl= TRUE)[[1]]) == 1 &&
%%              gregexpr(patLau,x,perl= TRUE)[[1]] == -1)         { return(0) }
%%          else { return(length(gregexpr(patLau,x,perl= TRUE)[[1]]))}
%%          } # get the number of [laughter]
%%getappNum <- function(x) {
%%          if (length(gregexpr(patApp,x,perl= TRUE)[[1]]) == 1 &&
%%              gregexpr(patApp,x,perl= TRUE)[[1]] == -1)         { return(0) }
%%          else { return(length(gregexpr(patApp,x,perl= TRUE)[[1]]))}
%%          } # get the number of [applause]
%%lauNum <- sapply(speechVec, getlauNum, USE.NAMES= FALSE)
%%appNum <- sapply(speechVec, getappNum, USE.NAMES= FALSE)
%%speechVec <- sapply(speechVec, function(x){iconv(x,from="WINDOWS-1251",to="UTF-8",sub = " ")})
%%speechVec <- sapply(speechVec, function(x){x <- gsub(patLau,"",x,perl= TRUE);x <- gsub(patApp,"",x,perl= TRUE)})
%%             # remove all the non-verbal tags from the speech text
%%names(speechVec) <- NULL # clean the speechVec name (otherwise, long and messy)
%% end.rcode


\subsection*{(d)}

The collection of speeches is stored in a clean fashion of list elements. The \textit{i}th entries of all the elements
in \textit{listSpeech} give information about the \textit{i}th speech. 
This is easy later for plotting variables changes over time.

%% begin.rcode eval=FALSE
%%listSpeech <- list()
%%listSpeech$id <- fileid
%%listSpeech$name <- namePres
%%listSpeech$date <- as.integer(dateTalk) # conversion for plotting, sorting
%%listSpeech$numLaughter <- lauNum
%%listSpeech$numApplause <- appNum
%%listSpeech$speech <- speechVec
%% end.rcode


\subsection*{(e) (f)}

Words and sentences are extracted from each speech, and are stored as individual elements of a (rather long)
character vector. Counts are also done on both words and sentences.

\newpage
%% begin.rcode eval=FALSE
%%# Speech analysis
%%# Grab words from the speech by replacing all non-word char with space and split
%%getWords <- function(x){x <- gsub("'","",x, perl= TRUE); x <- gsub('\\W+'," ",x, perl= TRUE);
%%                        xs <- unlist(strsplit(x, "[ ]+", perl= TRUE));return(xs[xs != ""])}
%%# Grab sentences from the speech by spliting with ending chars like [.!?]
%%getSents <- function(x){x <- gsub(" (Mr|Ms|Mrs|Dr|St|Sr|Jr)\\.","\\1",x, perl= TRUE);x <- gsub("[\\.!\\?][ \t]+","\n",x, perl= TRUE);
%%                        xs <- unlist(strsplit(x, "\n", perl= TRUE));return(xs[xs != ""])}
%%listSpeech$wc <- sapply(speechVec, function(x){return(length(getWords(x)))}, USE.NAMES= FALSE) # word count
%%listSpeech$sc <- sapply(speechVec, function(x){return(length(getSents(x)))}, USE.NAMES= FALSE) # sentence count
%%listSpeech$wMean <- sapply(speechVec, function(x){return(mean(nchar(getWords(x))))}, USE.NAMES= FALSE) #avg word length
%%listSpeech$wSD <- sapply(speechVec, function(x){return(sd(nchar(getWords(x))))}, USE.NAMES= FALSE) # word length sd
%%listSpeech$sMean <- sapply(speechVec, function(x){return(mean(nchar(getSents(x))))}, USE.NAMES= FALSE) # avg sentence length
%%listSpeech$sSD <- sapply(speechVec, function(x){return(sd(nchar(getSents(x))))}, USE.NAMES= FALSE) # sentence length sd
%% end.rcode



\subsection*{(g) (h)}

We now start to extract some features of interest from the speeches to analyze how the speeches have
changed over time. The result of all this is a list with each element containing the information
about a speech: the speech as a single string, the vector of sentences, the vector of words
and the additional quantification of variables about the speech from (g) as well as
the non-verbal variables from (c).

\begin{enumerate}
\item
Length in words and sentences \textit{wc,sc}
\item
Average and SD of word and sentence lengths \textit{wMean,wSD,sMean,sSD}
\item
Number of quotations in each speech, mean length (in words), and SD of length (in words) of the 
quotations in each speech \textit{quoNum,quoMean,quoSD}
\item
The most common meaningful words, where non-meaningful words are pre-defined \textit{cmw}
\item
Counts of the following words or word stems: 
\scriptsize
\begin{description}
\item[I, we]
\item[America{,n}]
\item[democra{cy,tic}]
\item[republic]
\item[Democrat{,ic}]
\item[Republican]
\item[free{,dom}]
\item[war]
\item[God] -- not including God bless
\item[God Bless]
\item[{Jesus, Christ, Christian}]
\item[Woman] -- I think would be interesting
\end{description}
\end{enumerate}
\normalsize

%% begin.rcode eval=FALSE
%% tmpList <- matrix(rep(0, 226*15),nrow=226,ncol=15)
%% ## Speech list with element-wise analysis
%% speechList <- list() #empty list
%% # Get the non-meaningful words file
%% system("wget -O 'cmw.txt' 'http://www.textfixer.com/resources/common-english-words.txt'")
%% commonWords <- readLines('cmw.txt', warn= FALSE)
%% commonWords <- unlist(strsplit(commonWords, ",", perl= TRUE))
%% # Element-wise analysis on each speech
%%for (i in 1:length(fileid)) {
%%	ss <- list() #empty list element
%%	# Global attributes: from results before (reproduce for better storage)
%%	ss$id <- fileid[i]
%%	ss$name <- namePres[i]
%%	ss$date <- as.integer(dateTalk[i])
%%	ss$numLaughter <- lauNum[i]
%%	ss$numApplause <- appNum[i]
%%	ss$speech <- speechVec[i]	
%%	# Indiv attributes: get from within each speech
%%	talkWords <- getWords(speechVec[i]) 
%%	talkSents <- getSents(speechVec[i])
%%	ss$words <- talkWords # words vector
%%	ss$sents <- talkSents # sentence vector	
%%	ss$wc <- length(talkWords) # word count
%%	ss$sc <- length(talkSents) # sentence count
%%	ss$wMean <- mean(nchar(talkWords)) # avg word length
%%	ss$wSD <- sd(nchar(talkWords)) # word length sd
%%	ss$sMean <- mean(nchar(talkSents)) # avg sentence length
%%	ss$sSD <- sd(nchar(talkSents)) # sentence length sd; ss[14]
%%	patQuo <- '\"(.*?)\"' # quotation pattern
%%	quo <- talkSents[grep(patQuo,talkSents,perl= TRUE)]
%%	if (length(quo) != 0) { # get quotation attributes
%%		quo <- sapply(quo, function(x){gsub(patQuo,"\\1",x)},USE.NAMES= FALSE)
%%		ss$quoNum <- length(quo)
%%		ss$quoMean <- mean(nchar(quo))
%%		ss$quoSD <- sd(nchar(quo))
%%	}
%%	else{
%%		ss$quoNum <- 0
%%		ss$quoMean <- 0
%%		ss$quoSD <- 0
%%	} #ss[17]
%%
%%	cmw <- sort(table(talkWords), decreasing = TRUE) 
%%	cmw <- cmw[which(!(names(cmw) %in% commonWords))] # get meaningful words
%%	ss$cmw <- cmw[cmw >= 10] #arbitrary cut-off for display
%%	ss$strIwe <- cmw[grep("^(I|[Ww]e)$",names(cmw))] #string 'I|We'; ss[19]
%%	ss$strAme <- cmw[grep("[Aa]merica(|n)",names(cmw))] #string 'America'
%%	ss$strDem <- cmw[grep("[Dd]emocra(cy|tic)",names(cmw))] #string 'democratic'
%%	ss$strRep <- cmw[grep("[Rr]epublic(|n)",names(cmw))] #string 'republican'
%%	ss$strFree <- cmw[grep("^[Ff]ree(|dom)$",names(cmw))] #string 'free'
%%	ss$strWar <- cmw[grep("^[Ww]ar(|s)$",names(cmw))] #string 'war'
%%	ss$strGod <- cmw[grep("^[Gg]od(|s)$",names(cmw))] #string 'God'
%%	ss$strChr <- cmw[grep("(Jesus|Christ|Christian)",names(cmw))] #string 'Chirst|Jesus'
%%	ss$strWoman <- cmw[grep("^[Ww]om[ae]n$",names(cmw))] #Mystring 'Woman'; ss[27]	
%%	ssGodBless <- talkSents[grep("[Gg]od [Bb]less",talkSents, perl= TRUE)]
%%	if (length(ssGodBless) != 0) { #string 'God Bless' from sentences
%%	ss$strGodBless <- sapply(ssGodBless, 
%%	                  function(x){
%%                    return(length(gregexpr("[Gg]od [Bb]less",x,perl=TRUE)[[1]]))}, 
%%	                  USE.NAMES= FALSE)
%%	}else{	ss$strGodBless <- 0	}
%%	# add to speechList and listSpeech
%%	speechList[[i]] <- ss
%%	tmpList[i, 1:3] <- unlist(ss[15:17]) #quo attr
%%	tmpList[i, 4:13] <- sapply(ss[19:28],sum) #cmw related attr
%%}
%%# prepare for plotting, store as listSpeech elements
%%listSpeech$quoNum <- tmpList[,1]
%%listSpeech$quoMean <- tmpList[,2]
%%listSpeech$quoSD <- tmpList[,3]
%%listSpeech$strIwe <- tmpList[,4]
%%listSpeech$strAme <- tmpList[,5]
%%listSpeech$strDem <- tmpList[,6]
%%listSpeech$strRep <- tmpList[,7]
%%listSpeech$strFree <- tmpList[,8]
%%listSpeech$strWar <- tmpList[,9]
%%listSpeech$strGod <- tmpList[,10]
%%listSpeech$strChr <- tmpList[,11]
%%listSpeech$strWoman <- tmpList[,12]
%%listSpeech$strGodBless <- tmpList[,13]
%% end.rcode


\subsection*{(i) (j)}

Some basic plots that show how the variables have changed over time are given below.

%% begin.rcode PS23-all, cache=TRUE, echo=FALSE, results="hide"
rm(list=ls(all=TRUE)) #remove all objects
#system("mkdir html")
setwd("./html")

# Download all html files
#system("wget -q -O 'index_pres.html' 'http://www.presidency.ucsb.edu/sou.php#axzz265cEKp1a'")
#system("fromdos index_pres.html")
indexPres <- readLines('index_pres.html',warn= FALSE)
# Get speech text source
patUrl1 <- '\\s{16}<td width=\\"\\d{2}\\" align=\\"center\\" class=\\"doclist\\"><a href=\\"'
indexPres <- indexPres[grep(patUrl1,indexPres,perl= TRUE)]
indexPres <- sapply(indexPres, function(x){gsub(patUrl1,"",x)},USE.NAMES= FALSE)
patUrl2 <- '\\">\\d{4}<\\/a>(\\*|)<\\/td>'
indexPres <- sapply(indexPres, function(x){gsub(patUrl2,"",x)},USE.NAMES= FALSE)
# Get file id from the source url
patUrl3 <- 'http:\\/\\/www\\.presidency\\.ucsb\\.edu\\/ws\\/index\\.php\\?pid='
fileid <- sapply(indexPres, function(x){gsub(patUrl3,"",x)},USE.NAMES= FALSE)
# Download all files and convert to unix
#sapply(1:length(fileid),
#       function(i){system(paste("wget -q -O '",fileid[i],".html' '",indexPres[i],"'",sep=""));
#                   system(paste("fromdos ",fileid[i],".html",sep=""))})

ff <- sapply(fileid, function(x){readLines(paste(x,".html",sep=""), warn= FALSE)})

# Get the president name
patName <- "^<title>(.*?)<\\/title>"
namePres <- ff[grep(patName,ff,perl= TRUE)]
namePres <- sapply(namePres, function(x){gsub(patName,"\\1",x)},USE.NAMES= FALSE)
namePres <- sapply(namePres, function(x){return(unlist(strsplit(x, ":"))[1])},USE.NAMES= FALSE)
# Get the talk date
patDate <- '^.*<span class=\\"docdate\\">(.*?)<\\/span>.*$'
dateTalk <- ff[grep(patDate,ff,perl= TRUE)]
dateTalk <- sapply(dateTalk, function(x){gsub(patDate,"\\1",x)},USE.NAMES= FALSE)
dateTalk <- sapply(dateTalk, function(x){gsub("^.*\\s","",x)},USE.NAMES= FALSE)
# Get the speech content text and prune for nice-format print
patText <- '^.*<span class=\\"displaytext\\">(.*?)<\\/span>.*$'
speechVec <- ff[grep(patText,ff,perl= TRUE)]
speechVec <- sapply(speechVec, function(x){gsub(patText,"\\1<p>",x)},USE.NAMES= FALSE) # grab speech text
speechVec <- sapply(speechVec, function(x){gsub("(<p.*?>|<\\/p>|<br>)","\n",x)},USE.NAMES= FALSE) # for line ending
speechVec <- sapply(speechVec, function(x){gsub("<.*?>","",x)},USE.NAMES= FALSE) # remove all html format
speechVec <- sapply(speechVec, function(x){x <- gsub("&mdash;"," -- ",x);
                                           x <- gsub("&nbsp;","  ",x);
                                           x <- gsub("&lsquo;"," ' ",x);
                                           x <- gsub("&#8226;"," \\. ",x);
                                           x <- gsub("&lt;"," < ",x);
                                           x <- gsub("&deg;"," degree ",x);
                                           x <- gsub("&pound;"," pound ",x);  
                                           x <- gsub("&fra.*?;"," 1/2 ",x);
                                           x <- gsub("&O.*?;","O",x);
                                           x <- gsub("&e.*?;","e",x);                                     
                                          },USE.NAMES= FALSE) # html special char
# Remove audience response tags (laughter & applause)
patLau <- '\\[.*?(Laughter|laughter).*?\\]'
patApp <- '\\[.*?(Applause|applause).*?\\]'
getlauNum <- function(x) {
          if (length(gregexpr(patLau,x,perl= TRUE)[[1]]) == 1 &&
              gregexpr(patLau,x,perl= TRUE)[[1]] == -1)         { return(0) }
          else { return(length(gregexpr(patLau,x,perl= TRUE)[[1]]))}
          }
getappNum <- function(x) {
          if (length(gregexpr(patApp,x,perl= TRUE)[[1]]) == 1 &&
              gregexpr(patApp,x,perl= TRUE)[[1]] == -1)         { return(0) }
          else { return(length(gregexpr(patApp,x,perl= TRUE)[[1]]))}
          }
lauNum <- sapply(speechVec, getlauNum, USE.NAMES= FALSE)
appNum <- sapply(speechVec, getappNum, USE.NAMES= FALSE)

speechVec <- sapply(speechVec, function(x){iconv(x,from="WINDOWS-1251",to="UTF-8",sub = " ")})
speechVec <- sapply(speechVec, function(x){x <- gsub(patLau,"",x,perl= TRUE);
                                           x <- gsub(patApp,"",x,perl= TRUE)})
names(speechVec) <- NULL
listSpeech <- list()
listSpeech$id <- fileid
listSpeech$name <- namePres
listSpeech$date <- as.integer(dateTalk)
listSpeech$numLaughter <- lauNum
listSpeech$numApplause <- appNum
listSpeech$speech <- speechVec

# Speech analysis
getWords <- function(x){x <- gsub("'","",x, perl= TRUE);
                        x <- gsub('\\W+'," ",x, perl= TRUE);
                        xs <- unlist(strsplit(x, "[ ]+", perl= TRUE))
                        return(xs[xs != ""])}
getSents <- function(x){x <- gsub(" (Mr|Ms|Mrs|Dr|St|Sr|Jr)\\.","\\1",x, perl= TRUE);
                        x <- gsub("[\\.!\\?][ \t]+","\n",x, perl= TRUE);
                        xs <- unlist(strsplit(x, "\n", perl= TRUE));
                        return(xs[xs != ""])}
listSpeech$wc <- sapply(speechVec, function(x){return(length(getWords(x)))}, USE.NAMES= FALSE)
listSpeech$sc <- sapply(speechVec, function(x){return(length(getSents(x)))}, USE.NAMES= FALSE)
listSpeech$wMean <- sapply(speechVec, function(x){return(mean(nchar(getWords(x))))}, USE.NAMES= FALSE)
listSpeech$wSD <- sapply(speechVec, function(x){return(sd(nchar(getWords(x))))}, USE.NAMES= FALSE)
listSpeech$sMean <- sapply(speechVec, function(x){return(mean(nchar(getSents(x))))}, USE.NAMES= FALSE)
listSpeech$sSD <- sapply(speechVec, function(x){return(sd(nchar(getSents(x))))}, USE.NAMES= FALSE)
tmpList <- matrix(rep(0, 226*15),nrow=226,ncol=15)

# Speech list with element-wise analysis
speechList <- list()
#system("wget -q -O 'common_words.txt' 'http://www.textfixer.com/resources/common-english-words.txt'")
commonWords <- readLines('common_words.txt', warn= FALSE)
commonWords <- unlist(strsplit(commonWords, ",", perl= TRUE))
for (i in 1:length(fileid)) {
	ss <- list()
	# Global attr
	ss$id <- fileid[i]
	ss$name <- namePres[i]
	ss$date <- as.integer(dateTalk[i])
	ss$numLaughter <- lauNum[i]
	ss$numApplause <- appNum[i]
	ss$speech <- speechVec[i]
	
	# Indiv attr
	talkWords <- getWords(speechVec[i]) 
	talkSents <- getSents(speechVec[i])
	ss$words <- talkWords
	ss$sents <- talkSents
	
	ss$wc <- length(talkWords)
	ss$sc <- length(talkSents)
	ss$wMean <- mean(nchar(talkWords))
	ss$wSD <- sd(nchar(talkWords))
	ss$sMean <- mean(nchar(talkSents))
	ss$sSD <- sd(nchar(talkSents)) #ss[14]
	
	patQuo <- '\"(.*?)\"'
	quo <- talkSents[grep(patQuo,talkSents,perl= TRUE)]
	if (length(quo) != 0) {
		quo <- sapply(quo, function(x){gsub(patQuo,"\\1",x)},USE.NAMES= FALSE)
		ss$quoNum <- length(quo)
		ss$quoMean <- mean(nchar(quo))
		ss$quoSD <- sd(nchar(quo))
	}
	else{
		ss$quoNum <- 0
		ss$quoMean <- 0
		ss$quoSD <- 0
	} #ss[17]
	
	cmw <- sort(table(talkWords), decreasing = TRUE)
	cmw <- cmw[which(!(names(cmw) %in% commonWords))]
	ss$cmw <- cmw[cmw >= 10] #arbitrary cut-off
	ss$strIwe <- cmw[grep("^(I|[Ww]e)$",names(cmw))] #ss[19]
	ss$strAme <- cmw[grep("[Aa]merica(|n)",names(cmw))]
	ss$strDem <- cmw[grep("[Dd]emocra(cy|tic)",names(cmw))]
	ss$strRep <- cmw[grep("[Rr]epublic(|n)",names(cmw))]
	ss$strFree <- cmw[grep("^[Ff]ree(|dom)$",names(cmw))]
	ss$strWar <- cmw[grep("^[Ww]ar(|s)$",names(cmw))]
	ss$strGod <- cmw[grep("^[Gg]od(|s)$",names(cmw))]
	ss$strChr <- cmw[grep("(Jesus|Christ|Christian)",names(cmw))]
	ss$strWoman <- cmw[grep("^[Ww]om[ae]n$",names(cmw))] #ss[27]
	
	ssGodBless <- talkSents[grep("[Gg]od [Bb]less",talkSents, perl= TRUE)]
	if (length(ssGodBless) != 0) {
	ss$strGodBless <- sapply(ssGodBless, 
	                         function(x){return(length(gregexpr("[Gg]od [Bb]less",x,perl=TRUE)[[1]]))}, 
	                         USE.NAMES= FALSE)
	}else{
	ss$strGodBless <- 0
	}
		
	# add to speechList and listSpeech
	speechList[[i]] <- ss
	tmpList[i, 1:3] <- unlist(ss[15:17]) #quo
	tmpList[i, 4:13] <- sapply(ss[19:28],sum) #cmw
}

listSpeech$quoNum <- tmpList[,1]
listSpeech$quoMean <- tmpList[,2]
listSpeech$quoSD <- tmpList[,3]
listSpeech$strIwe <- tmpList[,4]
listSpeech$strAme <- tmpList[,5]
listSpeech$strDem <- tmpList[,6]
listSpeech$strRep <- tmpList[,7]
listSpeech$strFree <- tmpList[,8]
listSpeech$strWar <- tmpList[,9]
listSpeech$strGod <- tmpList[,10]
listSpeech$strChr <- tmpList[,11]
listSpeech$strWoman <- tmpList[,12]
listSpeech$strGodBless <- tmpList[,13]
%% end.rcode


%% begin.rcode PS23-fig, echo=FALSE, results="hide", cache=TRUE, dev='cairo_pdf', fig.width=8, fig.height=5, fig.align="center"
# Create plots
attach(listSpeech)
plotTitle1 = "Presidential Speech Statistics Over Time"; xlab="Year"
# over time
layout(matrix(c(1,1,2,3), 2, 2, byrow = TRUE)); par(cex.main= 1)
typ="h"; col="blue"; lwd=6; 
plot(date,wc, type=typ,col=col,lwd=lwd, main=plotTitle1,xlab=xlab,ylab="Word Count")
typ="b";lwd=2;col="brown"
plot(date,wMean, type=typ,col=col,lwd=lwd, main=plotTitle1,xlab=xlab,ylab="Avg Word Length")
plot(date,wSD, type=typ,col=col,lwd=lwd, main=plotTitle1,xlab=xlab,ylab="Word Length Std Dev")
%% end.rcode

We can see that the word counts of the speeches are higher prior to 1920s than the modern ages.
So does the avg word length and word length sd. The presidents nowadays seem to like brevity.


Similar plots for sentence counts and quotation counts are given below. The similar trends stand.

%% begin.rcode PS23-fig2, echo=FALSE, results="hide", cache=TRUE, dev='cairo_pdf', fig.width=8, fig.height=5, fig.align="center"
layout(matrix(c(1,1,2,3), 2, 2, byrow = TRUE)); par(cex.main= 1)
typ="h"; col="blue"; lwd=6; 
plot(date,sc, type=typ,col=col,lwd=lwd, main=plotTitle1,xlab=xlab,ylab="Sentence Count")
typ="b";lwd=2;col="brown"
plot(date,sMean, type=typ,col=col,lwd=lwd, main=plotTitle1,xlab=xlab,ylab="Avg Sentence Length")
plot(date,sSD, type=typ,col=col,lwd=lwd, main=plotTitle1,xlab=xlab,ylab="Sentence Length Std Dev")

layout(matrix(c(1,1,2,3), 2, 2, byrow = TRUE)); par(cex.main= 1)
typ="h";lwd=6;col="blue"
plot(date,quoNum, type=typ,col=col,lwd=lwd, main=plotTitle1,xlab=xlab,ylab="Quotation Count")
typ="b";lwd=2;col="brown"
plot(date,quoMean, type=typ,col=col,lwd=lwd, main=plotTitle1,xlab=xlab,ylab="Avg Quotation Length")
plot(date,quoSD, type=typ,col=col,lwd=lwd, main=plotTitle1,xlab=xlab,ylab="Quotation Length Std Dev")
%% end.rcode

In recent years, with full records of non-verbal attributes, we can see that [Laughter] and [Applause] counts
have some positive correlation. The speeches receive more responses from the audience around 2000 than other years.


%% begin.rcode PS23-fig3, echo=FALSE, results="hide", cache=TRUE, dev='cairo_pdf', fig.width=8, fig.height=4.5, fig.align="center"
# recent years
plotTitle2 = "Presidential Speech Statistics in Modern U.S.A."
recY <- 1: max(c(which(numLaughter > 0),which(numApplause > 0)))
typ="h";lwd=6;col="blue"
par(mfrow=c(2,1),cex.main= 1)
plot(date[recY],numLaughter[recY], type=typ,col=col,lwd=lwd, main=plotTitle2,xlab=xlab,ylab="Laughter Count")
plot(date[recY],numApplause[recY], type=typ,col=col,lwd=lwd, main=plotTitle2,xlab=xlab,ylab="Applause Count")
%% end.rcode

And for presidents since Franklin Roosevelt in 1932, comparison between Republican presidents 
(Eisenhower, Nixon, Ford, Reagan, G. Bush, G.W. Bush) and Democratic presidents 
(Roosevelt, Truman, Kennedy, Johnson, Carter, Clinton, Obama) are also given below in the box-plot comparison.



%% begin.rcode PS23-figRD, echo=FALSE, results="hide", cache=TRUE, dev='cairo_pdf', fig.width=8, fig.height=12, fig.align="center"
# Rep vs Dem
plotTitle3 = "Republican vs. Democratic Presidential Speech Statistics (Since 1932)"
repPres <- c("Dwight D. Eisenhower", "Richard Nixon", "Gerald R. Ford", 
             "Ronald Reagan", "George Bush", "George W. Bush")
demPres <- c("Franklin D. Roosevelt", "Harry S. Truman", "John F. Kennedy", 
             "Lyndon B. Johnson", "Jimmy Carter", "William J. Clinton", "Barack Obama")
repY <- which(name %in% repPres)
demY <- which(name %in% demPres)
layout(matrix(1:16, 8, 2)); par(mar= c(1.5,6,1.5,6))
col="blue";cex= 0.6
boxplot(wc[repY],col=col,ylim=c(2000,10000)); legend("topleft", "Rep. Word Count",cex=cex)
boxplot(sc[repY],col=col,ylim=c(100,800)); legend("topleft", "Rep. Sentence Count",cex=cex)
boxplot(quoNum[repY],col=col,ylim=c(0,12)); legend("topleft", "Rep. Quotation Count",cex=cex)
boxplot(numLaughter[repY],col=col,ylim=c(0,8)); legend("topleft", "Rep. Laughter Count",cex=cex)
boxplot(numApplause[repY],col=col,ylim=c(0,4)); legend("topleft", "Rep. Applause Count",cex=cex)
boxplot(wMean[repY],col=col); legend("topleft", "Rep. Avg Word Length",cex=cex)
boxplot(sMean[repY],col=col,ylim=c(90,150)); legend("topleft", "Rep. Avg Sentence Length",cex=cex)
boxplot(quoMean[repY],col=col,ylim=c(0,400)); legend("topleft", "Rep. Avg Quotation Length",cex=cex)
col="brown";cex= 0.6
boxplot(wc[demY],col=col,ylim=c(2000,10000)); legend("topleft", "Dem. Word Count",cex=cex)
boxplot(sc[demY],col=col,ylim=c(100,800)); legend("topleft", "Dem. Sentence Count",cex=cex)
boxplot(quoNum[demY],col=col,ylim=c(0,12)); legend("topleft", "Dem. Quotation Count",cex=cex)
boxplot(numLaughter[demY],col=col,ylim=c(0,8)); legend("topleft", "Dem. Laughter Count",cex=cex)
boxplot(numApplause[demY],col=col,ylim=c(0,4)); legend("topleft", "Dem. Applause Count",cex=cex)
boxplot(wMean[demY],col=col); legend("topleft", "Dem. Avg Word Length",cex=cex)
boxplot(sMean[demY],col=col,ylim=c(90,150)); legend("topleft", "Dem. Avg Sentence Length",cex=cex)
boxplot(quoMean[demY],col=col,ylim=c(0,400)); legend("topleft", "Dem. Avg Quotation Length",cex=cex)
%% end.rcode

\newpage
Some additional research with plotting that illustrates how the speeches have changed over time are shown below.
The different keywords have evolved over time, with peak in WAR during wartime and GOD-rich in recent years.


%% begin.rcode PS23-ex1, echo=FALSE, results="hide", cache=TRUE, dev='cairo_pdf', fig.width=8, fig.height=9.5, fig.align="center"
# extra
par(mfrow=c(6,1),cex= 0.5, cex.main= 1)
typ="h"; col="blue"; lwd=6;
par(mar= c(0.5,6,4,6)); plot(date,strIwe, type=typ,col=col,lwd=lwd, main=plotTitle1,ylab="'I''We' Count")
par(mar= c(0.5,6,0.5,6)); plot(date,strAme, type=typ,col=col,lwd=lwd, ylab="'America' Count")
par(mar= c(0.5,6,0.5,6)); plot(date,strFree, type=typ,col=col,lwd=lwd, ylab="'Free/dom' Count")
par(mar= c(0.5,6,0.5,6)); plot(date,strWar, type=typ,col=col,lwd=lwd, ylab="'War' Count")
par(mar= c(0.5,6,0.5,6)); plot(date,strGod, type=typ,col=col,lwd=lwd, ylab="'God' Count") 
par(mar= c(4,6,0.5,6)); plot(date,strWoman, type=typ,col=col,lwd=lwd, xlab=xlab,ylab="'Woman' Count")
%% end.rcode

The comparison between Reps and Dems could be more interesting with the strings REPUBLICAN and DEMOCRATIC.
It seems that the Dems are more focused on their persuit while Reps are impartial to both words.

%% begin.rcode PS23-ex2, echo=FALSE, results="hide", cache=TRUE, dev='cairo_pdf', fig.width=5, fig.height=3.5, fig.align="center"
par(mfrow=c(2,2), mar= c(1,2,1,2)); cex=1
col="blue";
boxplot(strRep[repY],col=col,ylim=c(0,15)); legend("topleft", "Rep. 'Rep' Count",cex=cex)
col="purple"
boxplot(strDem[repY],col=col,ylim=c(0,15)); legend("topleft", "Rep. 'Dem' Count",cex=cex)
boxplot(strRep[demY],col=col,ylim=c(0,15)); legend("topleft", "Dem. 'Rep' Count",cex=cex)
col="brown";
boxplot(strDem[demY],col=col,ylim=c(0,15)); legend("topleft", "Dem. 'Dem' Count",cex=cex)
%% end.rcode


Some additional variables that quantify speech in interesting ways are produced. The number of digits shown within the
speech is a good indicator whether the president speaks more quantatively, or vice versa, more qualitatively.

%% begin.rcode PS23-ex3, echo=FALSE, results="hide", cache=TRUE, dev='cairo_pdf', fig.width=8, fig.height=4.5, fig.align="center"
getDigts <- function(x){x <- gsub("(\\d)[,\\.](\\d)","\\1\\2",x, perl= TRUE);
                        x <- gsub("\\D","\n",x, perl= TRUE);
                        xs <- unlist(strsplit(x, "\n", perl= TRUE));
                        return(xs[xs != ""])}
listSpeech$dc <- sapply(speechVec, function(x){return(length(getDigts(x)))}, USE.NAMES= FALSE)
typ="h"; col="blue"; lwd=6; 
plot(date,listSpeech$dc, type=typ,col=col,lwd=lwd, main=plotTitle1,xlab=xlab,ylab="Digits Count")
%% end.rcode



\end{document}
