\documentclass{article}

\usepackage{color}
\definecolor{dkgreen}{rgb}{0,0.6,0}
\definecolor{gray}{rgb}{0.5,0.5,0.5}
\definecolor{mauve}{rgb}{0.58,0,0.82}
\usepackage[margin=1in]{geometry}
\usepackage{fancyhdr}
\pagestyle{fancy}
\lhead{\today}
\chead{Ying Qiao SID:21412301}
\rhead{Stat243 Fa12: Problem Set 3}
\lfoot{}
\cfoot{\thepage}
\rfoot{}
\usepackage{graphicx}
\usepackage{textcomp}
\usepackage{lmodern}
\usepackage[T1]{fontenc}
\usepackage{listings}

%% for inline R code: if the inline code is not correctly parsed, you will see a message
\newcommand{\rinline}[1]{SOMETHING WORNG WITH knitr}
%% begin.rcode setup, include=FALSE
% opts_chunk$set(fig.path='figure/latex-', cache.path='cache/latex-')
%% end.rcode


\begin{document}
\section*{Problem 1}

Within \textit{mixtureMean.RData}, there are two test cases (A with large number of components \textit{K}; 
and B with small \textit{K}) consisting of a vector of $\mu$ values, a list of weights and a list of IDs that
map the weights to the corresponding components in the mean vector.

\subsubsection*{(a)}
One line code using \textit{sapply()} that will calculate the weighted mean $\sum_{k=1}^{m_i} w_{i,k}\mu_{ID_{i,k}}$.

%% begin.rcode PS31-0, cache=TRUE, results="hide"
%%#### Problem Set 3
%%#### 1. mixtureMean
library(rbenchmark)
rm(list = ls(all = TRUE)) # remove all objects
load('mixtureMean.RData') # import data
%% end.rcode

%% begin.rcode, cache=FALSE, eval=FALSE, results="hide"
%%# (a) original data storage
%%mixmeanA <- sapply(1:length(IDsA), function (i){return(sum(muA[IDsA[[i]]]*wgtsA[[i]]))}) # sapply test case A
%%mixmeanB <- sapply(1:length(IDsB), function (i){return(sum(muB[IDsB[[i]]]*wgtsB[[i]]))}) # sapply test case B
%% end.rcode

\subsubsection*{(b)}
I set up the objects under case A as two matrices, one storing the $\mu$ values used for calcuation and one
storing the corresponding weights for those mean values. The size of the these two matrices are the same,
which is $n \times max(m_i)$, i.e. the number of observations times the maximum number of components per observation.

%% begin.rcode PS31-b, cache=TRUE, results="hide"
%%# (b) data setup for A: K=1000
%%# table out as matrix storing the mu[ids]/wgts in cols for each observation row
idnum <- length(IDsA)
idlen <- sapply(1:idnum, function(i){return(length(IDsA[[i]]))})
maxmi <- max(idlen)
muidA <- matrix(as.numeric(NA), nr = idnum, nc = maxmi)
for (i in 1:idnum) {
	muidA[i, 1:idlen[i] ] <- muA[IDsA[[i]]]
}
wtidA <- matrix(as.numeric(NA), nr = idnum, nc = maxmi)
for (i in 1:idnum) {
	wtidA[i, 1:idlen[i] ] <- wgtsA[[i]]
}
mixmeanA2 <- rowSums(muidA*wtidA, na.rm = TRUE)
%% end.rcode

\subsubsection*{(c)}
The set up of data objects under case B is even simpler. As $K=10$ which is small, I just use a $n \times K$ matrix to
store all the weights for the observations of the mean vector, leaving the untouched components with weight 0.

%% begin.rcode PS31-c, cache=TRUE, results="hide"
%%# (c) data setup for B: K=10
%%# small K can allow us to store all the IDs as truth table for each u
idnum <- length(IDsB)
munum <- length(muB) # K is small
wtidB <- matrix(0, nr = munum, nc = idnum)
for (i in 1:idnum) {
	tmpwt <- rep(0, munum)
	tmpwt[ IDsB[[i]] ] <- wgtsB[[i]]
	wtidB[ ,i] <- tmpwt
}
mixmeanB2 <- colSums(muB*wtidB)
%% end.rcode

\subsubsection*{(d)}
The comparison for the two test cases are shown below using benchmarking functions.

%% begin.rcode PS31-d, cache=TRUE, results="markup", tidy = FALSE
%%# (d) efficiency comparison
# (d) efficiency comparison
benchmark(A1 = {mixmeanA <- sapply(1:length(IDsA), 
                            function(i){ return( sum(muA[IDsA[[i]]]*wgtsA[[i]]) ) })},
          A2 = {mixmeanA2 <- rowSums(muidA*wtidA, na.rm = TRUE)}, 
          replications = 5)
all.equal(mixmeanA, mixmeanA2)
benchmark(B1 = {mixmeanB <- sapply(1:length(IDsB), 
                            function(i){ return( sum(muB[IDsB[[i]]]*wgtsB[[i]]) ) })},
          B2 = {mixmeanB2 <- colSums(muB*wtidB)}, 
          replications = 5)
all.equal(mixmeanB, mixmeanB2)
%% end.rcode

We can see that there is about 50X to 100X speed-up compared to the original case, which shows the 
advantages of re-arranging the data objects.

\newpage
\section*{Problem 2}
The CSC (compressed sparse column) format for storing sparse matrices has three components:
\small
\begin{description}
\item \textit{values}: a vector of non-zero entries, stored with column major ordering
\item \textit{rowIndices}: a vector of row indices, one for each non-zero entry
\item \textit{colPointers}: a vector of entry index at which each column starts (length of $ncol + 1$)
\end{description}
\normalsize

%% begin.rcode PS32-0, cache=TRUE, results="hide", message=FALSE
%%#### Problem Set 3
%%#### 2. CSC matrix 
library(Matrix) # sparse matrix package
library(compiler)
library(rbenchmark)
rm(list = ls(all = TRUE)) # remove all objects
source("cscFromC.R")
%% end.rcode

\subsubsection*{(a)}
I re-write the C-style \textit{makeCSC()} function as below.

%% begin.rcode, cache=FALSE, results="markup"
%%### (a) R version of makeCSC
makeCSCr <- function(matT){
	matCSC = list()
	dimInfo <- which(matT != 0, arr.ind = TRUE) # array indices info (row, col)
	matCSC$values <- matT[dimInfo] #non-zero matrix entries 
	matCSC$rowIndices <- dimInfo[ ,1] # row indices
	matCSC$colPointers <- c(1, cumsum(tabulate(dimInfo[ ,2], nbins = ncol(matT)))+1 ) # cumsum the num of entries in each col
  return(matCSC)
}
m <- matrix(c(1,0,0,7,0,2,0,0,0,0,0,0,0,0,0,4), nr=4)
makeCSCr(m)
%% end.rcode

\subsubsection*{(b)}
I used the profiling tool to assess the computation bottleneck of my function, which is shown below.
I generated the test matrices using the \textit{makeTestMatrix()} function provided in \textit{cscFromC.R}.

%% begin.rcode PS32-b, cache=TRUE, results="markup"
m <- makeTestMatrix(2500)
Rprof("makeCSCr.prof", interval = 0.01)
system.time(mr <- makeCSCr(m))
Rprof(NULL)
summaryRprof("makeCSCr.prof")$by.self
%% end.rcode

So, an improved version is shown below with built-in packages.
%% begin.rcode, cache=FALSE, results="markup"
%%### (b) # more efficient after Rprof() and improvement in coding
makeCSCr2 <- function(matT){
	matCSC = list()
	M <- as(matT, "dgCMatrix")
	matCSC$values <- M@x #non-zero matrix entries 
	matCSC$rowIndices <- M@i + 1 # row indices
	matCSC$colPointers <- M@p + 1 # cumsum the num of entries in each col
  return(matCSC)
}
m <- matrix(c(1,0,0,7,0,2,0,0,0,0,0,0,0,0,0,4), nr=4)
makeCSCr2(m)
%% end.rcode

\subsubsection*{(c)}
The speed test for C-style, original R-style, package-based R-style and byte-compiled R is shown below.
I have also made a comparison function for checking the accuracy of the functions written in different styles.
%% begin.rcode PS32-c, cache=TRUE, results="markup"
compMat <- function(m1, m2){ # compare the CSC representation of matrix
	flag = all.equal(m1$values, m2$values)
	flag = flag && all.equal(m1$rowIndices, m2$rowIndices)
	flag = flag && all.equal(m1$colPointers, m2$colPointers)
	return(flag)
}
makeCSCrCMP <- cmpfun(makeCSCr) # byte compiled
%%############## Test ##################
m <- makeTestMatrix(6400)

system.time(mc <- makeCSC(m))
system.time(mr <- makeCSCr(m))
compMat(mc, mr)

%%######## benchmark ##########
benchmark(makeCSC(m), makeCSCr(m), makeCSCr2(m), makeCSCrCMP(m), replications = 2,
          columns = c("test", "replications", "elapsed", "relative", "user.self", "sys.self"))
%% end.rcode

We can see that the pakage-based R-style code runs the fastest, that byte-compiled version runs slightly
faster than the original one and that C-style code runs the slowest.

\subsubsection*{(d)}
In my R-style \textit{makeCSCr()} function, there is no additional copy of the full matrix or other matrices
of that same size. The result is shown below with the \textit{gc()} function.

%% begin.rcode PS32-d, cache=TRUE, results="markup"
%%### (d) memory usage
gc() #initial
m <- makeTestMatrix(2500)
gc() # with m
system.time(mr <- makeCSCr(m))
gc() # after mr
%% end.rcode


\newpage
\section*{Problem 3}
We would like to set the lower triangle of a matrix to equal the transpose of the upper triangle.

%% begin.rcode PS33-0, cache=TRUE, results="hide", message=FALSE
%%#### Problem Set 3
%%#### 3. lower.triangle(mat) = transpose(upper.triangle(mat)) 
rm(list = ls(all = TRUE)) # remove all objects
m <- matrix(1:25, nr = 5) # simple test matrix
%% end.rcode

\subsubsection*{(a)}
The code shown below does not work. As R uses column-major ordering to store matrix entries, when extracted
using \textit{upper.tri()}, the ordering is messed up as shown below inline with the diagnosis code.

%% begin.rcode PS33-a, cache=FALSE, results="markup"
m <- matrix(1:25, nr = 5) # simple test matrix
m #showcase
%%### (a)
m[lower.tri(m)] = t(m[upper.tri(m)]) # this code does not work
m #showcase
# should be row-major conversion here for substitution
m[upper.tri(m)] # convert to col-major vector     
%% end.rcode

\subsubsection*{(b)}
The working code is shown below with inline explaination of how it works. The main idea is to extract the 
extries from the transposed matrix to perserve the row-major ordering of the original matrix.

%% begin.rcode PS33-b, cache=FALSE, results="markup"
%%### (b)
m <- matrix(1:25, nr = 5) # simple test matrix
m #debug
m[upper.tri(m)] #debug, col-major
%%### after examination, we have to get row-major upper tirangle
t(m) # transpose of orignal m
t(m)[lower.tri(m)] # col-major of t(m) = row-major of m
%%### the working one-line code to do the job
m #showcase
m[lower.tri(m)] <- t(m)[lower.tri(m)] # this code works
m #showcase
%% end.rcode

\subsubsection*{(c)}
The \textit{lower.tri<-} replacement function is shown below. The optional \textit{byrow} argument is 
similar to that in \textit{matrix()}, which let the function to fill the martix with row-major ordering when TRUE.

%% begin.rcode PS33-c, cache=FALSE, results="markup", tidy = FALSE
%%### (c)
`lower.tri<-` <- function(x, byrow = FALSE, value){ 
    #replacement function for lower.tri
    
    x <- as.matrix(x)
    
    if ( length(x[lower.tri(x)]) != length(value) ) 
      stop("Vector length mismatch!")
	
    if (!byrow)	x[lower.tri(x)] <- value # col-major
    else { # row-major
        tmpx <- t(x) # no replacement function for t()
        tmpx[upper.tri(tmpx)] <- value
        x <- t(tmpx)
    }
    return(x)
}

%%########test#########
m <- matrix(1:25, nr = 5) # simple test matrix
m
lower.tri(m) <- 1:10 ; 	m
lower.tri(m, byrow = TRUE) <- 1:10 ; m
lower.tri(m) <- c(1,2,3)
%% end.rcode


\newpage
\section*{Problem 4}
For the object-oriented coding to generate a Markov chain with a fixed number of states, I choose to use the 
widely accepted and fast speed S3 class. A simple test case is generated with matrix attributes required for a 
Markov chain object.

%% begin.rcode PS34-0, cache=TRUE, results="hide", message=FALSE
%%#### Problem Set 3
%%#### 4. Markov Chain (OOP)
library(methods)
rm(list = ls(all = TRUE)) # remove all objects
rnd <- rnorm(50000) # rnd sample
rnd <- rnd[rnd > 0] # prob > 0

p = 5 # num of states
x0 <- rep(0, p) # get length p vector
x0[1] <- 1 # initialize with starting from state 1
xm <- matrix(sample(rnd, p*p), nr = p) # get p-by-p matrix of rnds
xm <- xm / rowSums(xm) # normalize rows

drawState <- function (p, pr) {
    if (length(pr) != p) stop("Vector length mismatch!")
    states <- diag(nrow = p, ncol = p) # state vectors
    id <- sample(1:5, 1, prob = pr)
    return (as.vector(states[id, ])) # get random states according to prob=pr
}
%% end.rcode

For an object in  Markov chain class, it has the initial state vector \textit{init}
and two matrices i.e. transition matrix \textit{P} and chain matrix \textit{chain} which stores the process. 
In details, the transition matrix $P$ of dimension $p \times p$ gives the transition probability
$P[i,j] = Pr(X_n = j | X_{n-1} = i)$. 

\subsubsection*{(a)}
The constructor that creates an \textit{Markov} class object utlizes a method \textit{runSteps()} to run
the chain for a given number of steps requested by user. The validity is checked inline with the code shown below.

%% begin.rcode PS34-a, cache=FALSE, results="markup", tidy = FALSE
%%### (a) create the Markov S3 class:
%%### initial state, transition matrix, num of steps, current state and [if needed, chain];

%%### creating a "runSteps" method for the MarkovS3 class construction
runSteps <- function(object, ...) UseMethod("runSteps") # generic method

runSteps.MarkovS3 <- function(obj) { 
    if (obj$n == 0) {
        obj$chain <- matrix(obj$init, nr = 1)
        return (obj)  # no more steps, simple return the initial state
    } else {
        p = length(obj$init)
        obj$chain <- matrix(as.numeric(NA), nr = obj$n + 1, nc = p)
        obj$chain[1, ] <- obj$init
        for (i in 1:obj$n)
            obj$chain[i + 1, ] <- drawState(p, as.vector(obj$chain[i, ] %*% obj$P))
    }
    return(obj)
}

%%### constructor for 'MarkovS3' class
MarkovS3 <- function(init = NA, P = NA, n = 0, chain = NA){	
    # check validity
    p = length(init) # num of states
    nr = dim(P)[1]; nc = dim(P)[2]; # dim of transition matrix
	
    if (n < 0) 
        stop("Need positive integer steps!")
    if (p < 2)
        stop("Need at least two states for the Markov chain!")
    if (nr != nc) 
        stop("The transistion matrix has to be square!")
    if (p != nr)
        stop("Mismatch between state vector length and transition matrix dimension!")
    if (length(init[init < 0]) > 0) 
        stop("All state probabilities should be positive!")
    if (length(P[P < 0]) > 0) 
        stop("All transition probabilities should be positive!")
    if (sum(init) != 1) 
        stop("State probabilities should sum to one!")
    if (!all.equal(rowSums(P), rep(1, p)))
        stop("Transition probabilities for each state should sum to one!")

    # initialize the Markov chain
    obj <- list(init = init, P = P, n = n, chain = NA)
    class(obj) <- 'MarkovS3'

    # construct the Markov chain
    obj <- runSteps(obj)
    return(obj)
}

%% ### simple tests
mkc0 <- MarkovS3(x0, xm)
mkc1 <- MarkovS3(x0, xm, n = 1)
mkc1
mkc2 <- MarkovS3(x0, xm, n = 1000)
%% end.rcode

\subsubsection*{(b)}
Three operators ('+', '-', '[') for the Markov chain class is developed as shown below.

%% begin.rcode PS34-b, cache=FALSE, results="markup", tidy = FALSE
%%#### (b) Markov chain S3 class-specific operators
`+.MarkovS3` <- function(obj, incr) {
    # check validity
    if (incr < 0) stop("Need positive integer steps!")
    if (incr == 0) return (obj)
        
    p = length(obj$init)
    addchain <- matrix(as.numeric(NA), nr = incr + 1, nc = p)
    addchain[1, ] <- obj$chain[obj$n + 1, ] # current state
    for (i in 1:incr) 
        addchain[i + 1, ] <- drawState(p, as.vector(addchain[i, ] %*% obj$P))
    
    obj$n <- obj$n + as.integer(incr) # add steps
    obj$chain <- rbind(obj$chain, addchain[2:(incr+1), ])
    return(obj)
}

`-.MarkovS3` <- function(obj, decr) {
    # check validity
    if (decr < 0) stop("Need positive integer steps!")
    if (obj$n < decr) stop("Not enough steps of states to remove!")
    if (decr == 0) return (obj)
	
    obj$n <- obj$n - as.integer(decr) # remove steps
    obj$chain <- obj$chain[1:(obj$n + 1), ]
    if (obj$n == 0) 
        obj$chain <- matrix(obj$chain, nr = 1)
    return(obj)
}

`[.MarkovS3` <- function(obj, idVec) {
    # check validity
    if (length(idVec) == 0) stop("No indices for extracting chain states!")
    if (length(idVec[idVec < 1 | idVec > obj$n+1])) stop("Indices out of bound!")	

    return(obj$chain[idVec, ]) # init state is 1st
}

%%# show only the chain
mkc1$chain
(mkc1 + 1)$chain
(mkc1 - 1)$chain
mkc2[10:15]
%% end.rcode

\subsubsection*{(c)}
Three methods, i.e. \textit{plot()}, \textit{summary()}, \textit{print()} or \textit{show()} 
that produce nice, formatted output are developed for Markov chain class shown below. 

%% begin.rcode PS34-c, cache=TRUE, results="markup", tidy=FALSE, dev='cairo_pdf', fig.width=8, fig.height=5.5, fig.align="center"
%% #### (c) Markov chain S3 class-specific functions
plot.MarkovS3 <- function(obj, ...) {
    if (obj$n == 0)
        cat("The current state is the same as the initial state\n", obj$chain[1, ], "\n")
    if (obj$n > 0){
        data <- which(obj$chain > 0, arr.ind = TRUE)
        plot(data[ ,1], data[ ,2], xlab="Steps", ylab="States Prob", ...)
        hist(data[ ,2], c(0:5+0.5), probability = TRUE, 
             main = "Histogram of states", xlab = "Markov chain states")
    }
}

summary.MarkovS3 <- function(obj, ...) {
    p <- length(obj$init)
    allPre <- colSums(obj$chain[-(obj$n+1), ]) # previous states counts
    Tsub <- obj$chain[-1, ] - obj$chain[-(obj$n+1), ] # transition records (1:from,-1:to)
    Tadd <- obj$chain[-1, ] + obj$chain[-(obj$n+1), ] 
    Tadd[which(Tadd == 1)] <- 0 # retaining the state
    Pemp <- diag(colSums(Tadd)/2/allPre, nrow = p, ncol = p) #retaining states counts, diag
    for (i in 1:p)
        for (j in 1:p)
            if (i != j) 
                Pemp[i,j] <- length(which(Tsub[ ,i] == 1 & Tsub[ ,j] == -1))/allPre[i]

    cat("The initial state of the Markov chain is\n"); print(obj$init)
    cat("The transition matrix of this chain is\n"); print(obj$P)
    cat("The current step is", obj$n, ", and the current state is\n"); 
    print(obj$chain[obj$n+1, ])
    cat("The empirical transition probabilities are \n"); print(Pemp)
    cat("The empirical state probabilities are \n"); print(colMeans(obj$chain))
}


print.MarkovS3 <- function(obj, ...) {
    cat("Initial state\n"); print(obj$init)
    cat("Transition matrix\n"); print(obj$P)
    cat("Current step\n"); print(obj$n)
    cat("Current state\n"); print(obj$chain[obj$n+1, ])
}

summary(mkc2)
print(mkc1)
plot(mkc0)
plot(mkc2)
%% end.rcode


\end{document}
