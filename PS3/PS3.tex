\documentclass{article}

\usepackage{color}
\definecolor{dkgreen}{rgb}{0,0.6,0}
\definecolor{gray}{rgb}{0.5,0.5,0.5}
\definecolor{mauve}{rgb}{0.58,0,0.82}
\usepackage[margin=1in]{geometry}
\usepackage{fancyhdr}
\pagestyle{fancy}
\lhead{\today}
\chead{Ying Qiao SID:21412301}
\rhead{Stat243 Fa12: Problem Set 3}
\lfoot{}
\cfoot{\thepage}
\rfoot{}
\usepackage{graphicx}
\usepackage{textcomp}
\usepackage{lmodern}
\usepackage[T1]{fontenc}
\usepackage{listings}

%% for inline R code: if the inline code is not correctly parsed, you will see a message
\newcommand{\rinline}[1]{SOMETHING WORNG WITH knitr}
%% begin.rcode setup, include=FALSE
% opts_chunk$set(fig.path='figure/latex-', cache.path='cache/latex-')
%% end.rcode


\begin{document}
\section*{Problem 1}

Within \textit{mixtureMean.RData}, there are two test cases (A with large number of components \textit{K}; 
and B with small \textit{K}) consisting of a vector of $\mu$ values, a list of weights and a list of IDs that
map the weights to the corresponding components in the mean vector.

\subsection*{(a)}
One line code using \textit{sapply()} that will calculate the weighted mean $\sum_{k=1}^{m_i} w_{i,k}\mu_{ID_{i,k}}$.

%% begin.rcode PS31-0, cache=TRUE, results="hide"
%%#### Problem Set 3
%%#### 1. mixtureMean
%%library(rbenchmark)
%%rm(list = ls(all = TRUE)) # remove all objects
%%load('mixtureMean.RData') # import data
%% end.rcode

%% begin.rcode, cache=FALSE, eval=FALSE, results="hide"
%%# (a) original data storage
%%mixmeanA <- sapply(1:length(IDsA), function (i){return(sum(muA[IDsA[[i]]]*wgtsA[[i]]))}) # sapply test case A
%%mixmeanB <- sapply(1:length(IDsB), function (i){return(sum(muB[IDsB[[i]]]*wgtsB[[i]]))}) # sapply test case B
%% end.rcode

\subsection*{(b)}
I set up the objects under case A as two matrices, one storing the $\mu$ values used for calcuation and one
storing the corresponding weights for those mean values. The size of the these two matrices are the same,
which is $n \times max(m_i)$, i.e. the number of observations times the maximum number of components per observation.

%% begin.rcode PS31-b, cache=TRUE, results="hide"
%%# (b) data setup for A: K=1000
%%# table out as matrix storing the mu[ids]/wgts in cols for each observation row
idnum <- length(IDsA)
idlen <- sapply(1:idnum, function(i){return(length(IDsA[[i]]))})
maxmi <- max(idlen)

muidA <- matrix(as.numeric(NA), nr = idnum, nc = maxmi)
for (i in 1:idnum) {
	muidA[i, 1:idlen[i] ] <- muA[IDsA[[i]]]
}
wtidA <- matrix(as.numeric(NA), nr = idnum, nc = maxmi)
for (i in 1:idnum) {
	wtidA[i, 1:idlen[i] ] <- wgtsA[[i]]
}
mixmeanA2 <- rowSums(muidA*wtidA, na.rm = TRUE)
%% end.rcode

\subsection*{(c)}
The set up of data objects under case B is even simpler. As $K=10$ which is small, I just use a $n \times K$ matrix to
store all the weights for the observations of the mean vector, leaving the untouched components with weight 0.

%% begin.rcode PS31-c, cache=TRUE, results="hide"
%%# (c) data setup for B: K=10
%%# small K can allow us to store all the IDs as truth table for each u
idnum <- length(IDsB)
munum <- length(muB) # K is small

wtidB <- matrix(0, nr = munum, nc = idnum)
for (i in 1:idnum) {
	tmpwt <- rep(0, munum)
	tmpwt[ IDsB[[i]] ] <- wgtsB[[i]]
	wtidB[ ,i] <- tmpwt
}
mixmeanB2 <- colSums(muB*wtidB)
%% end.rcode

\subsection*{(d)}
The comparison for the two test cases are shown below using benchmarking functions.

%% begin.rcode PS31-d, cache=TRUE, results="markup"
%%# (d) efficiency comparison
# (d) efficiency comparison
benchmark(A1 = {mixmeanA<-sapply(1:length(IDsA), function(i){return(sum(muA[IDsA[[i]]]*wgtsA[[i]]))})},
          A2 = {mixmeanA2 <- rowSums(muidA*wtidA, na.rm = TRUE)}, replications = 5)
all.equal(mixmeanA, mixmeanA2)

benchmark(B1 = {mixmeanB<-sapply(1:length(IDsB), function(i){return(sum(muB[IDsB[[i]]]*wgtsB[[i]]))})},
          B2 = {mixmeanB2 <- colSums(muB*wtidB)}, replications = 5)
all.equal(mixmeanB, mixmeanB2)
%% end.rcode

We can see that there is about 50X to 70X speed-up compared to the original case, which shows the 
advantages of re-arranging the data objects.

\newpage
\section*{Problem 2}











\end{document}
